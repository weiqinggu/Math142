\documentclass[12pt,letterpaper]{hmcpset}
\usepackage[margin=1in]{geometry}
\usepackage{graphicx}
\usepackage{amsthm}
\usepackage{enumitem}
\usepackage{amsmath, cancel}

\input{macros.tex}

% info for header block in upper right hand corner
\name{Joseph Gardi}
\class{Differential Geometry}
\assignment{Notes}
\duedate{Monday, September 30 2019}

\renewcommand{\labelenumi}{{(\alph{enumi})}}


\begin{document}
\underline{Preliminary Definitions}
\begin{itemize}
\item A neighborhood around a point $p$ is $\{ x: ||x - p|| \leq r\}$ for some
  radius $r$. This set is shaped like a ball and denotoed $B_r(p)$.
\item If we say something is true locally around $x$ we mean it is true for all
  points in some sufficiently small neighborhood around $x$.
\item A homeomorphism is a continuous function with a coninuous inverse.
\item Intuition for continuous funcions. If $f$ is continuous then $a$ close to
  $b$ implies that $f(a)$ close $f(b)$. To illustrate consider my body. My wrist
  is close to my hand. So if we apply a continuous transformationsto the position
  of each of my body parts theen my wrist will still be close to my hand. All
  the movements I normally do are continuous transformations.
  But if I cut off my own hand then my hand won't be close to my wrist anymore.
  That is a discontinuous transformation. There is a discontinuous jump from my
  wrist after cutting to my hand after cutting.
\item The differential matrix for a function $f(x, y)$:
  \begin{align*}
    Df \triangleq \begin{bmatrix}
    \frac{\partial f_1}{\partial x} & \frac{\partial f_1}{\partial y} \\
    \frac{\partial f_2}{\partial x} & \frac{\partial f_2}{\partial y}  
    \end{bmatrix}
  \end{align*}
\end{itemize}
It will also be helpful to remember the first order taylor expansion for a
continuous differentiable function $f: R^2 \rightarrow R^2$,
\begin{align*}
  f(x, y) = f(u, v) + Df(u, v)\begin{bmatrix}x - u \\ x - v\end{bmatrix}
\end{align*}
\underline{Theorem:} Inverse function theorem in high dimension \\
Let $\rho: (u, v) \rightarrow (x(u, v), y(u, v))$ be a differentiable function. \\
$\rho$ is locally (i.e. for some sufficiently small neighborhood) invertable around
$(a, b)$ if and only if $D\rho((a, b))$ is an
invertable matrix. Recall,
\begin{align*}
  D\rho =
  \begin{bmatrix}
    \frac{\partial x}{\partial u} & \frac{\partial x}{\partial v} \\
    \frac{\partial y}{\partial u} & \frac{\partial y}{\partial v}  
  \end{bmatrix}
\end{align*}
Also, $\rho^{-1}$ is differentiable.  
\begin{align*}

\end{align*}

We will first define regular surface. Then we take characteristic properties of
regular surfaces to define manifolds. Manifolds are like a generalization of
regular surfaces. \\
We will put the reimennian matric on each tangent space. \\

This is similar to how we generalized $R^3$ to metric spaces in analysis.

Suppose we are interested in the miotion of a particle. Then we take, as the
state oof the particle, the pair of 3 dimensional vectoros $(x, v)$ where $x$ is
the position and $v$ is the velocity. If we know the particle must stay on a
sphere $M$ then we it follows that $v$ must always be tangent to $M$. Then our
state space $S$ is not all pairs of 3-vectors but he tangent bundle of $M$ which
is a manifold.
\begin{align*}
  S = \{ (x, v) : v\text{ is tangent to }M\}
\end{align*}
\underline{Definition:} $S \subset R^3$ is a regular surface if for each $p \in S$ there
exists a neighborhood $V$ in $R^3$ and a map $x: U \rightarrow V \bigcap S$ of an open set $U
\subset R^2$ onto $V \bigcap S \subset R^3$ such that,
\begin{itemize}
\item $x$ is differentiable (so we can use calculus)
\item $x$ is a homeomoorphism (so we can use analysis)
\item $x$ is regular (so we can use linear algebra). Since $x$ is regular there
  is a tangent plane at each point in $S$.
\end{itemize}

Suppose we have a regular surface $V \subseteq R^3$ with a mapping $x: (u, v) \rightarrow (x(u, v)
y(u, v), z(u, v))$. Then the tangent plane at a point $q$ is spanned by the
columns of $Dx_q = \begin{bmatrix}\frac{\partial x}{\partial u} & \frac{\partial x}{\partial v} \\
\frac{\partial y}{\partial u} & \frac{\partial y}{\partial v} \\
\frac{\partial z}{\partial u} & \frac{\partial z}{\partial v} \\
\end{bmatrix}$.  \\
\underline{Notation:}
\begin{align*}
  \frac{\partial(x, y)}{\partial(u, v)} \triangleq \begin{bmatrix}\frac{\partial x}{\partial u} & \frac{\partial x}{\partial v} \\
    \frac{\partial u}{\partial u} & \frac{\partial y}{\partial v}
  \end{bmatrix}
\end{align*}
$\{(x, y, f(x, y)) : x, y \in R \}$ is a surface.\\
The parameterization for this surface is $a(x, y) = (x, y, f(x, y))$ \\
Now we willl prove his is a regular surface. $da = \begin{bmatrix}1 ^ 0 \\ 0 &
  1 \\ \frac{\partial f}{\partial x} & \frac{\partial f}{\partial y}\end{bmatrix}$. This is invertable so
$a$ has a continuous inverse. So it is homeomorphic.
\underline{Definition:} For a differentible function $f: U \subset R^3 \rightarow R$. 
\end{document}
