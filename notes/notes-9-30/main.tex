\documentclass[12pt,letterpaper]{hmcpset}
\usepackage[margin=1in]{geometry}
\usepackage{graphicx}
\usepackage{amsthm}
\usepackage{enumitem}
\usepackage{amsmath, cancel}

% Theorems
\usepackage{amsthm}
\renewcommand\qedsymbol{$\blacksquare$}
\makeatletter
\@ifclassloaded{article}{
    \newtheorem{definition}{Definition}[section]
    \newtheorem{example}{Example}[section]
    \newtheorem{theorem}{Theorem}[section]
    \newtheorem{corollary}{Corollary}[theorem]
    \newtheorem{lemma}{Lemma}[theorem]
}{
}
\makeatother

% Random Stuff
\setlength\unitlength{1mm}

\newcommand{\insertfig}[3]{
\begin{figure}[htbp]\begin{center}\begin{picture}(120,90)
\put(0,-5){\includegraphics[width=12cm,height=9cm,clip=]{#1}}\end{picture}\end{center}
\caption{#2}\label{#3}\end{figure}}

\newcommand{\insertxfig}[4]{
\begin{figure}[htbp]
\begin{center}
\leavevmode \centerline{\resizebox{#4\textwidth}{!}{\input
#1.pstex_t}}
\caption{#2} \label{#3}
\end{center}
\end{figure}}

\long\def\comment#1{}

\newcommand\abs[1]{\left\lvert#1\right\rvert}
\newcommand\norm[1]{\left\lVert#1\right\rVert}
\DeclareMathOperator*{\argmin}{arg\,min}
\DeclareMathOperator*{\argmax}{arg\,max}

% bb font symbols
\newfont{\bbb}{msbm10 scaled 700}
\newcommand{\CCC}{\mbox{\bbb C}}

\newfont{\bbf}{msbm10 scaled 1100}
\newcommand{\CC}{\mbox{\bbf C}}
\newcommand{\PP}{\mbox{\bbf P}}
\newcommand{\RR}{\mbox{\bbf R}}
\newcommand{\QQ}{\mbox{\bbf Q}}
\newcommand{\ZZ}{\mbox{\bbf Z}}
\renewcommand{\SS}{\mbox{\bbf S}}
\newcommand{\FF}{\mbox{\bbf F}}
\newcommand{\GG}{\mbox{\bbf G}}
\newcommand{\EE}{\mbox{\bbf E}}
\newcommand{\NN}{\mbox{\bbf N}}
\newcommand{\KK}{\mbox{\bbf K}}
\newcommand{\KL}{\mbox{\bbf KL}}

% Vectors
\renewcommand{\aa}{{\bf a}}
\newcommand{\bb}{{\bf b}}
\newcommand{\cc}{{\bf c}}
\newcommand{\dd}{{\bf d}}
\newcommand{\ee}{{\bf e}}
\newcommand{\ff}{{\bf f}}
\renewcommand{\gg}{{\bf g}}
\newcommand{\hh}{{\bf h}}
\newcommand{\ii}{{\bf i}}
\newcommand{\jj}{{\bf j}}
\newcommand{\kk}{{\bf k}}
\renewcommand{\ll}{{\bf l}}
\newcommand{\mm}{{\bf m}}
\newcommand{\nn}{{\bf n}}
\newcommand{\oo}{{\bf o}}
\newcommand{\pp}{{\bf p}}
\newcommand{\qq}{{\bf q}}
\newcommand{\rr}{{\bf r}}
\renewcommand{\ss}{{\bf s}}
\renewcommand{\tt}{{\bf t}}
\newcommand{\uu}{{\bf u}}
\newcommand{\ww}{{\bf w}}
\newcommand{\vv}{{\bf v}}
\newcommand{\xx}{{\bf x}}
\newcommand{\yy}{{\bf y}}
\newcommand{\zz}{{\bf z}}
\newcommand{\0}{{\bf 0}}
\newcommand{\1}{{\bf 1}}

% Matrices
\newcommand{\Ab}{{\bf A}}
\newcommand{\Bb}{{\bf B}}
\newcommand{\Cb}{{\bf C}}
\newcommand{\Db}{{\bf D}}
\newcommand{\Eb}{{\bf E}}
\newcommand{\Fb}{{\bf F}}
\newcommand{\Gb}{{\bf G}}
\newcommand{\Hb}{{\bf H}}
\newcommand{\Ib}{{\bf I}}
\newcommand{\Jb}{{\bf J}}
\newcommand{\Kb}{{\bf K}}
\newcommand{\Lb}{{\bf L}}
\newcommand{\Mb}{{\bf M}}
\newcommand{\Nb}{{\bf N}}
\newcommand{\Ob}{{\bf O}}
\newcommand{\Pb}{{\bf P}}
\newcommand{\Qb}{{\bf Q}}
\newcommand{\Rb}{{\bf R}}
\newcommand{\Sb}{{\bf S}}
\newcommand{\Tb}{{\bf T}}
\newcommand{\Ub}{{\bf U}}
\newcommand{\Wb}{{\bf W}}
\newcommand{\Vb}{{\bf V}}
\newcommand{\Xb}{{\bf X}}
\newcommand{\Yb}{{\bf Y}}
\newcommand{\Zb}{{\bf Z}}

% Calligraphic
\newcommand{\Ac}{{\cal A}}
\newcommand{\Bc}{{\cal B}}
\newcommand{\Cc}{{\cal C}}
\newcommand{\Dc}{{\cal D}}
\newcommand{\Ec}{{\cal E}}
\newcommand{\Fc}{{\cal F}}
\newcommand{\Gc}{{\cal G}}
\newcommand{\Hc}{{\cal H}}
\newcommand{\Ic}{{\cal I}}
\newcommand{\Jc}{{\cal J}}
\newcommand{\Kc}{{\cal K}}
\newcommand{\Lc}{{\cal L}}
\newcommand{\Mc}{{\cal M}}
\newcommand{\Nc}{{\cal N}}
\newcommand{\Oc}{{\cal O}}
\newcommand{\Pc}{{\cal P}}
\newcommand{\Qc}{{\cal Q}}
\newcommand{\Rc}{{\cal R}}
\newcommand{\Sc}{{\cal S}}
\newcommand{\Tc}{{\cal T}}
\newcommand{\Uc}{{\cal U}}
\newcommand{\Wc}{{\cal W}}
\newcommand{\Vc}{{\cal V}}
\newcommand{\Xc}{{\cal X}}
\newcommand{\Yc}{{\cal Y}}
\newcommand{\Zc}{{\cal Z}}

% Bold greek letters
\newcommand{\alphab}{\hbox{\boldmath$\alpha$}}
\newcommand{\betab}{\hbox{\boldmath$\beta$}}
\newcommand{\gammab}{\hbox{\boldmath$\gamma$}}
\newcommand{\deltab}{\hbox{\boldmath$\delta$}}
\newcommand{\etab}{\hbox{\boldmath$\eta$}}
\newcommand{\lambdab}{\hbox{\boldmath$\lambda$}}
\newcommand{\epsilonb}{\hbox{\boldmath$\epsilon$}}
\newcommand{\nub}{\hbox{\boldmath$\nu$}}
\newcommand{\mub}{\hbox{\boldmath$\mu$}}
\newcommand{\zetab}{\hbox{\boldmath$\zeta$}}
\newcommand{\phib}{\hbox{\boldmath$\phi$}}
\newcommand{\psib}{\hbox{\boldmath$\psi$}}
\newcommand{\thetab}{\hbox{\boldmath$\theta$}}
\newcommand{\taub}{\hbox{\boldmath$\tau$}}
\newcommand{\omegab}{\hbox{\boldmath$\omega$}}
\newcommand{\xib}{\hbox{\boldmath$\xi$}}
\newcommand{\sigmab}{\hbox{\boldmath$\sigma$}}
\newcommand{\pib}{\hbox{\boldmath$\pi$}}
\newcommand{\rhob}{\hbox{\boldmath$\rho$}}

\newcommand{\Gammab}{\hbox{\boldmath$\Gamma$}}
\newcommand{\Lambdab}{\hbox{\boldmath$\Lambda$}}
\newcommand{\Deltab}{\hbox{\boldmath$\Delta$}}
\newcommand{\Sigmab}{\hbox{\boldmath$\Sigma$}}
\newcommand{\Phib}{\hbox{\boldmath$\Phi$}}
\newcommand{\Pib}{\hbox{\boldmath$\Pi$}}
\newcommand{\Psib}{\hbox{\boldmath$\Psi$}}
\newcommand{\Thetab}{\hbox{\boldmath$\Theta$}}
\newcommand{\Omegab}{\hbox{\boldmath$\Omega$}}
\newcommand{\Xib}{\hbox{\boldmath$\Xi$}}

% mixed symbols
\newcommand{\sinc}{{\hbox{sinc}}}
\newcommand{\diag}{{\hbox{diag}}}
\renewcommand{\det}{{\hbox{det}}}
\newcommand{\trace}{{\hbox{tr}}}
\newcommand{\tr}{\trace}
\newcommand{\sign}{{\hbox{sign}}}
\renewcommand{\arg}{{\hbox{arg}}}
\newcommand{\var}{{\hbox{var}}}
\newcommand{\cov}{{\hbox{cov}}}
\renewcommand{\Re}{{\rm Re}}
\renewcommand{\Im}{{\rm Im}}
\newcommand{\eqdef}{\stackrel{\Delta}{=}}
\newcommand{\defines}{{\,\,\stackrel{\scriptscriptstyle \bigtriangleup}{=}\,\,}}
\newcommand{\<}{\left\langle}
\renewcommand{\>}{\right\rangle}
\newcommand{\Psf}{{\sf P}}
\newcommand{\T}{\top}
\newcommand{\m}[1]{\begin{bmatrix} #1 \end{bmatrix}}


% info for header block in upper right hand corner
\name{Joseph Gardi}
\class{Differential Geometry}
\assignment{Notes}
\duedate{Monday, September 30 2019}

\renewcommand{\labelenumi}{{(\alph{enumi})}}


\begin{document}
\underline{Preliminary Definitions}
\begin{itemize}
\item A neighborhood around a point $p$ is $\{ x: ||x - p|| \leq r\}$ for some
  radius $r$. This set is shaped like a ball and denotoed $B_r(p)$.
\item If we say something is true locally around $x$ we mean it is true for all
  points in some sufficiently small neighborhood around $x$.
\item A homeomorphism is a continuous function with a coninuous inverse.
\item Intuition for continuous funcions. If $f$ is continuous then $a$ close to
  $b$ implies that $f(a)$ close $f(b)$. To illustrate consider my body. My wrist
  is close to my hand. So if we apply a continuous transformationsto the position
  of each of my body parts theen my wrist will still be close to my hand. All
  the movements I normally do are continuous transformations.
  But if I cut off my own hand then my hand won't be close to my wrist anymore.
  That is a discontinuous transformation. There is a discontinuous jump from my
  wrist after cutting to my hand after cutting.
\item The differential matrix for a function $f(x, y)$:
  \begin{align*}
    Df \triangleq \begin{bmatrix}
    \frac{\partial f_1}{\partial x} & \frac{\partial f_1}{\partial y} \\
    \frac{\partial f_2}{\partial x} & \frac{\partial f_2}{\partial y}  
    \end{bmatrix}
  \end{align*}
\end{itemize}
It will also be helpful to remember the first order taylor expansion for a
continuous differentiable function $f: R^2 \rightarrow R^2$,
\begin{align*}
  f(x, y) = f(u, v) + Df(u, v)\begin{bmatrix}x - u \\ x - v\end{bmatrix}
\end{align*}
\underline{Theorem:} Inverse function theorem in high dimension \\
Let $\rho: (u, v) \rightarrow (x(u, v), y(u, v))$ be a differentiable function. \\
$\rho$ is locally (i.e. for some sufficiently small neighborhood) invertable around
$(a, b)$ if and only if $D\rho((a, b))$ is an
invertable matrix. Recall,
\begin{align*}
  D\rho =
  \begin{bmatrix}
    \frac{\partial x}{\partial u} & \frac{\partial x}{\partial v} \\
    \frac{\partial y}{\partial u} & \frac{\partial y}{\partial v}  
  \end{bmatrix}
\end{align*}
Also, $\rho^{-1}$ is differentiable.  
\begin{align*}

\end{align*}

We will first define regular surface. Then we take characteristic properties of
regular surfaces to define manifolds. Manifolds are like a generalization of
regular surfaces. \\
We will put the reimennian matric on each tangent space. \\

This is similar to how we generalized $R^3$ to metric spaces in analysis.

Suppose we are interested in the miotion of a particle. Then we take, as the
state oof the particle, the pair of 3 dimensional vectoros $(x, v)$ where $x$ is
the position and $v$ is the velocity. If we know the particle must stay on a
sphere $M$ then we it follows that $v$ must always be tangent to $M$. Then our
state space $S$ is not all pairs of 3-vectors but he tangent bundle of $M$ which
is a manifold.
\begin{align*}
  S = \{ (x, v) : v\text{ is tangent to }M\}
\end{align*}
\underline{Definition:} $S \subset R^3$ is a regular surface if for each $p \in S$ there
exists a neighborhood $V$ in $R^3$ and a map $x: U \rightarrow V \bigcap S$ of an open set $U
\subset R^2$ onto $V \bigcap S \subset R^3$ such that,
\begin{itemize}
\item $x$ is differentiable (so we can use calculus)
\item $x$ is a homeomoorphism (so we can use analysis)
\item $x$ is regular (so we can use linear algebra). Since $x$ is regular there
  is a tangent plane at each point in $S$.
\end{itemize}

Suppose we have a regular surface $V \subseteq R^3$ with a mapping $x: (u, v) \rightarrow (x(u, v)
y(u, v), z(u, v))$. Then the tangent plane at a point $q$ is spanned by the
columns of $Dx_q = \begin{bmatrix}\frac{\partial x}{\partial u} & \frac{\partial x}{\partial v} \\
\frac{\partial y}{\partial u} & \frac{\partial y}{\partial v} \\
\frac{\partial z}{\partial u} & \frac{\partial z}{\partial v} \\
\end{bmatrix}$.  \\
\underline{Notation:}
\begin{align*}
  \frac{\partial(x, y)}{\partial(u, v)} \triangleq \begin{bmatrix}\frac{\partial x}{\partial u} & \frac{\partial x}{\partial v} \\
    \frac{\partial u}{\partial u} & \frac{\partial y}{\partial v}
  \end{bmatrix}
\end{align*}
$\{(x, y, f(x, y)) : x, y \in R \}$ is a surface.\\
The parameterization for this surface is $a(x, y) = (x, y, f(x, y))$ \\
Now we willl prove his is a regular surface. $da = \begin{bmatrix}1 ^ 0 \\ 0 &
  1 \\ \frac{\partial f}{\partial x} & \frac{\partial f}{\partial y}\end{bmatrix}$. This is invertable so
$a$ has a continuous inverse. So it is homeomorphic.
\underline{Definition:} For a differentible function $f: U \subset R^3 \rightarow R$. 
\end{document}
