\documentclass[12pt,letterpaper]{hmcpset}
\usepackage[margin=1in]{geometry}
\DeclareMathOperator{\Tr}{Tr}
\usepackage{graphicx}
\usepackage{amsthm}
\usepackage{enumitem}
\usepackage{amsmath, cancel}

\input{macros.tex}

% info for header block in upper right hand corner
\name{Joseph Gardi}
\class{Differential Geometry}
\assignment{Notes}
\duedate{Monday, Nov 11th 2019}

\renewcommand{\labelenumi}{{(\alph{enumi})}}


\begin{document}
\underline{Statical manifold} \\
Gaussians form a manifold. You can use differential geometry to crerate a
distance function between gaussians. Euclidean distance doesn't work for this.
\\
\underline{3 approaches to big data problems} \\
1) Supervised learning: Assume a parametric model and then find parameters that
minimize some loss function that measures the error rate on some labeled data set. \\
2) Probabalistic aprooach: Maximize likelihood of the data \\
3) Geometric method: Geometric methods let us use visual intuition. For the
linear regression problem want to make $y \approx X\theta$ where $y$ is the correct outputs
vector, $X$ is an input matrix and $\theta$ is the vector with the parameters. $X\theta$
is some linear combination of the columns of $X$. Suppose that $X$ is $n$ by
$m$. The possible outputs for $X\theta$ is $B = span(\text{columns of }X)$. That is an
$m$ dimensional subspace of $R^n$ with the columns of $X$ as it's basis. But $y$ lives in $R^n$. So we have to find the point in
$B$ closest to $y$. You can do that by projecting $y$ onto $B$. \\
Look at the rate of change of The frame $\{x_u, x_v, N\}$ to detect how the
manifold is curved. \\
\underline{Christoffel symbols} \\
$x_{uu} = \Gamma^{11}_1x_u + \Gamma^{11}_2x_v + L_1N$ \\
$x_{uv} = \Gamma^{12}_1x_u + \Gamma^{12}_2x_v + L_2N$ \\
$x_{vu} = \Gamma^{21}_1x_u + \Gamma^{21}_2x_v + L_2N$ \\
$x_{vv} = \Gamma^{22}_1x_u + \Gamma^{22}_2x_v + L_2N$ \\
$N_u = a_{11}x_u + a_{21}x_v$ \\
$N_v = a_{12}x_u + a_{22}x_v$ \\
Then $e=L_1, f=L_2, g=L_3$. \\
\underline{How to find christoffel symbols} \\
Take inner product of both sides with $x_u$ and $x_v$ for the first 4 equations
for the christoffel symbols. This gives you a system of equations.
\begin{align*}
  <x_{uu}, x_u> &= \Gamma^{11}_1<x_u, x_u> + \Gamma^{11}_2<x_v, x_u> \\
  <x_{uu}, x_v> &= \Gamma^{11}_1<x_u, x_v> + \Gamma^{11}_2<x_v, x_v> \\
  &\vdots
\end{align*}
Solve that system of equations to find the christoffeel symbols.\\
\underline{Theorem} \\
The gaussian curvature ofa surface is invariatnt by local isometries.
\end{document}
