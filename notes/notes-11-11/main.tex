\documentclass[12pt,letterpaper]{hmcpset}
\usepackage[margin=1in]{geometry}
\DeclareMathOperator{\Tr}{Tr}
\usepackage{graphicx}
\usepackage{amsthm}
\usepackage{enumitem}
\usepackage{amsmath, cancel}

\input{macros.tex}

% info for header block in upper right hand corner
\name{Joseph Gardi}
\class{Differential Geometry}
\assignment{Notes}
\duedate{Monday, Nov 4th 2019}

\renewcommand{\labelenumi}{{(\alph{enumi})}}


\begin{document}
\underline{A big picture of geometry of gauss map} \\\\
\underline{Motivation: } We want to use maps and their differentials to study
the surfaces. What kind of maps should we consider? $Gauss\; Map: S \rightarrow S^2, p \rightarrow
N(p)$. Let $p$ be a point on a surface $S$ and let $\mathbf{x}_u, \mathbf{x}_v$
be a basis for $T_p(S)$. Then
\begin{align*}
  N(p) = \frac{\mathbf{x}_u \times \mathbf{x}_v}{||\mathbf{x}_u \times \mathbf{x}_v||}
\end{align*}
\underline{The second fundamental form}
$II_p(v) = -<dN_p(v), v>$\\
2) We can diagonalize $dN_p$. Let $k_1, k_2$ be the eigenvalues. Recall that the
eigenvalues are the principle curvatures. That means there exists an orthonormal basis
$\{e_1. e_2\} \in T_p(S)$ such that,
\begin{align*}
  dN_p(e_1) = k_1e_1 \\
  dN_p(e_2) = k_2e_2
\end{align*}
$-k_1, -k_2$ are the max and min of $\{II_p(v) : v \in T_p(S) \}$. Each invariant
characteristic of $dN_p$ has geometric meaning.
\begin{enumerate}
\item  $II_p(v)i \triangleq \text{normal curvature along v}$
\item $k_1, k_2 \triangleq \text{principal curvature}$
\item $det(dN_p) \triangleq \text{Gaussian curvature}$
\item $-\frac{1}{2}\; tr(dN_p)$
\end{enumerate}
\underline{The Gauss Map in local coordinates} \\
Let $S$ be a surface.
Let $\mathbf{x}(u, v)$ be a paramaeterization at a point $p \in S$. \\
Then \textbf{local coordinates} are $u, v$.
A neighborhood in local coordinates gets mapped to a neighborhood on the
surface. \\
Let $\alpha$ be a regular curve. Then $dN_p(\alpha'(0)) = N_u u'(0) + N_v v'(0)$. \\
Note that $N_u, N_v \in T_p(S)$. Then $N_u=\begin{bmatrix}\mathbf{x}_u &
  \mathbf{x}_v\end{bmatrix}\begin{bmatrix}a_{11} & a_{12}\\a_{21} &
  a_{22}\end{bmatrix}$. Then that
matrix representation of $dN_p$ in local coordinates is $A$.  \\
\underline{The second fundamental form in local coordinates} \\
Let $S$ be a surface. Let $w = \alpha'(0)$ where $\alpha$ is a regular curve in $S$. Let $p =
\alpha(0) \in S$. Then,
\begin{align*}
  II_p(w) &= II_p(\alpha'(0)) \\
  &\triangleq -<dN_p(w), w> \\
  &= -u'(0)^2<N_u, \mathbf{x}_u> - u'(0)v'(0)(<N_u, \mathbf{x}_v> + <N_v, \mathbf{x}_u>) - v'(0)^2<N_v, \mathbf{x}_v>
\end{align*}
So $e=-<N_u, \mathbf{x}_u>, f=-<N_u, \mathbf{x}_v>, g=-<N_v, \mathbf{x}_v>$. \\
Calculating $e, f, g$ is hard. Recall that $<N, \mathbf{x}_u> = 0$. Therefore,
$<N_u, \mathbf{x}_u> + <N, \mathbf{x}_{uu} = 0$. Therefore, $-<N_u,
\mathbf{x}_u> = <N, \mathbf{x}_{uu}>$. So then $e = <N, \mathbf{x}_{uu}>, f=-<N,
\mathbf{x}_{vu}>, g=<N, \mathbf{x}_{vv}>$. So all you need is some derivatives,
dot products, and cross products. \\
\underline{Claim} \\
\begin{align*}
  \begin{bmatrix}a_{11} & a_{12}\\
a_{21} & a_{22}
\end{bmatrix} = -\begin{bmatrix}e & f \\ f & g\end{bmatrix} \begin{bmatrix}E & F
\\ F & G\end{bmatrix}^{-1}
\end{align*}
\underline{Proof:} \\
$-e = <N_u, \mathbf{x}_u> = <a_{11}\mathbf{x}_u + a_{21}\mathbf{x}_v,
\mathbf{x}_u> = a_{11}<\mathbf{x}_u, \mathbf{x}_u> a_{21}<\mathbf{x}_u,
\mathbf{x}_v> = a_{11}E + a_{21}F$. Similarly, $-f = a_{11}F + a_{21}G,
-g=a_{12}F + a_{22}G$. This matches the given equation.\\
The principal curvatures are the eigenvalues of $dN_p$ so they are the roots of
the polynomial $k^2 + k(a_{11} + a_{12}) + \det(dN_p)$. \\
$F=0$ if and only if $\mathbf{x}_u \perp \mathbf{x}_v$. \\
If $F=0$ then $k_1 = e/E, k_2=g/G$. For any compact manifold at a point $p$ there exists a
parameterization such that $\mathbf{x}_u \perp \mathbf{x}_v$. \\
\begin{align*}
  x(u, v) &= \begin{bmatrix}
    (a + r\cos u) \cos v \\
    (a + r\cos u) \sin v \\
    r\sin u
    \end{bmatrix} \\
  \mathbf{x}_u &=
  \begin{bmatrix}
    -r\sin u \cos v \\
    -r \sin u \sin v \\
    r\cos u
  \end{bmatrix} \\
  \mathbf{x}_v &=
  \begin{bmatrix}
    -(a + r\cos u) \sin v \\
    (a + r \cos u) \cos v \\
    0
  \end{bmatrix} \\
  \mathbf{x}_u \times \mathbf{x}_v &= \begin{bmatrix}-r\cos u (a + r \cos u) \cos v\\-r\cos u (a + r\cos u) \sin v\\-r\sin u \cos^2v  (a + r \cos
    u) - r \sin u \sin^2 v (a + r\cos u)\end{bmatrix} \\
                              &= \begin{bmatrix}r\cos u (a + r \cos u) \cos v\\-r\cos u (a + r\cos u) \sin v\\-r\sin u  (a + r \cos
    u) \end{bmatrix}\\
                              &= -r(a + r\cos u)\begin{bmatrix}\cos u \cos v\\\cos u \sin v\\\sin u \end{bmatrix} \\
  N &= \frac{\mathbf{x}_u \times \mathbf{x}_v}{||\mathbf{x}_u \times \mathbf{x}_v||}
= \begin{bmatrix}\cos u \cos v\\\cos u \sin v\\\sin u 
    \end{bmatrix} \frac{1}{\cos^2u\cos^2v + \cos^2 u \sin^2 v +
  \sin^2u} \\
                              &= \begin{bmatrix}\cos u \cos v\\\cos u \sin v\\\sin u\end{bmatrix} \\
  \mathbf{x}_{uu} &= \begin{bmatrix}-r\cos u \cos v \\
    -r \cos u \sin v \\
    -r \sin u
  \end{bmatrix} \\
  \mathbf{x}_{vu} &= \begin{bmatrix}
    r\sin u \sin v \\
    -r\sin u \cos v \\
    0
  \end{bmatrix} \\
  \mathbf{x}_{vv} &= \begin{bmatrix}
    -(a + r\cos u) \cos v \\
    -(a + r\cos u) \sin v \\
    0
  \end{bmatrix} \\
  e &= <N, \mathbf{x}_{uu}> \\
          &= -r \cos^2 u \cos^2 v - r \cos^2 u \sin^2 v - r \sin^2 u \\
          &= -r \\
  f &= <N, \mathbf{x}_{vu}> \\
          &= r \sin u \sin v \cos u \cos v - r \sin u \cos v \cos u \sin v \\
          &= 0 \\
            g &= <N, \mathbf{x}_{vv}> \\
          &= -(a + r\cos u)(\cos^2 v \cos u + \sin^2 v \cos u) \\
          &= -(a + r \cos u) \cos u \\
  E &= <\mathbf{x}_u, \mathbf{x}_u> = r^2 \sin^2 u \cos^2 v + r^2 \sin^2 u \sin^2 v + r^2 \cos^2 u \\
\end{align*}
\begin{align*}
          &= 1\\
  F &= <\mathbf{x}_u, \mathbf{x}_v> = 0 \\
  G &= <\mathbf{x}_v, \mathbf{x}_v> = (a + r \cos u)^2 
\end{align*}
Then the curvature is,
\begin{align*}
  \frac{\left(a + r \cos{\left(u \right)}\right)^3 \cos{\left(u
\right)}}{r}
  \end{align*}
\end{document}
