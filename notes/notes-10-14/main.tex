\documentclass[12pt,letterpaper]{hmcpset}
\usepackage[margin=1in]{geometry}
\usepackage{graphicx}
\usepackage{amsthm}
\usepackage{enumitem}
\usepackage{amsmath, cancel}

\input{macros.tex}

% info for header block in upper right hand corner
\name{Joseph Gardi}
\class{Differential Geometry}
\assignment{Notes}
\duedate{Monday, October 14th 2019}

\renewcommand{\labelenumi}{{(\alph{enumi})}}


\begin{document}
$\rho$ is differentable at $p$, if $y^{-1} \circ \rho \circ x$ is differentible and well
defined. So this means compositions of differentible functions are differentible.\\\\

The differential map is a linear map. It maps $span([1, 0, 0], [0, 1, 0])$ to
the tangent plane. We can use it to change into the local cordinates system. \\

Consider a vector space $V$ of finite dimension. then having an inner product on
$V$ is equivalent to having a norm on $V$. $||v + w||^2 = <v+w, v+w> = <v, v> +
2<v, w> + <w, w> \implies <v, w> = ||v+w||^2 - ||v||^2 - ||w||^2$.
We've already shown in a previous lecture that we can get a norm from an inner
product. Now we've shown that we can get an inner product from a norm. So
they're equivalent. \\\\

Now we put a measurement on a manifold. We call a manifold with a measurement on
it a riemannian manifold. For a reegular surface, the measurement is just the
norm squared. \\\\

\underline{First fundamental form} \\
Anything that eats a vector and spits out a number is a form.
We can use the first fundamental form to find lengths and angles on the
manifold.

The \textbf{first fundamental form}  of $\mathbb{R}^3 \supset S$ induces on each
tangent plane $T_p(S)$ of a regualr surface $S$ an inner product, to be denoted
by $< , >_p$. If $w_1, w_2 \in T_p(S) \subset \mathbb{R}^3$, then $<w_1, w_2>$ is equal
to the innner product of $w_1, w_2$ as vectors in $\mathbb{R}^3$Since this inner
product is a symmetric bilinear form therre is a corresponding quadratic form
$I_p: T_p(S) \rightarrow \mathbb{R}$ given by,
\begin{align*}
  I_p(w) = <w, w>_p = ||w||^2 \geq 0
\end{align*}
The quadratic fom $I_p$ on $T_p(S)$ defined above is called the firsst
fundamental form of the regular surface $S$ at $p \in S$. 

\underline{Second fundamental form} \\
Let $S'$ be the set of curves tangent to some $v \in S$.  $k_1$ is defined as the
largeset curvature of any curve in $S'$. $k_2$ is defined as the smallest
curvature of any curve in $S'$.
Gaussian cuvature is defind as $k_1k_2$. Meean curvature iss $\frac{k_1 + k_2}{2}$

The second fundamental form is equivalent to gaussian curvature in
$\mathbb{R}^3$. \\\\

\underline{How to find first fundmanet form in local coordinates} \\
Given $w \in T_p(S)$, find a curve $\alpha(t)$ on $S$ such that $\alpha(t) = x(u(t), v(t))$
with $\alpha'(t) = x_uu'(t) + x_v v'(t)$ , $\alpha'(0) = x_u u'(0) + x_v v'(0) = [x_u,
x_v]\begin{bmatrix}u'(0)\\v'(0)\end{bmatrix}$\\
Then,
\begin{align*}
  I_p(w) &= ||w||^2 \\
  &= <u'(0)x_u + v'(0)x_v, u'(0)x_u + v'(0)x_v> \\
  &= u'(0)^2<x_u, x_u> + 2u'(0)v'(0)<x_u, x_v> + v'(0)^2<x_v, x_v> \\
  &= [u'(0), v'(0)]\begin{bmatrix}<x_u, x_u> & <x_u, x_v>\\<x_u, x_v>&<x_v, x_v>\end{bmatrix}[u'(0), v'(0)]
\end{align*}
$E=<x_u, x_u>, F=<x_u, x_v>, G=<x_v, x_v>$ are called the first fundamental
cooeficients. \\
We can generlize this to a manifold. Let $n$ be the dimension of the manifold. Let $p$ be a point on the manifold. Let
$\{u_1, \cdots, u_{n-1}\}$ be a basis for the tangent space at $p$. Then the
riemmani metric on the manifold is,
\begin{align*}
  ||w||_{riemmani}^2 = w^T \begin{bmatrix}<x_{u_1}, x_{u_1}> & \cdots & <x_{u_1}, x_{u_{n-1}}>\\
    & \ddots & \\
    <x_{u_{n-1}}, x_{u_1}> & \cdots & <x_{u_{n-1}}, x_{u_{n-1}}>
\end{bmatrix}
\end{align*}
Angle at which two curves intersect can be calculated with,
\begin{align*}
  \cos \theta = \frac{<\alpha'(t_0), \beta'(t_0)>}{||\alpha'(t_0)|| ||\beta'(t_0)||}
\end{align*}
The area of a bounded region $R \in S$ on a regular surface $S$ with
parameterization $x$ is,
\begin{align*}
  \int\int_{x^{-1}(R)} ||x_u \times x_v|| du dv = \int\int_{x^{-1}(R)} \sqrt{EG - F} du dv
\end{align*}
$x_u = (-r \sin u \cos v, -r \sin u \sin v, r \cos u)$\\
$x_v = (-(a + r \cos u) \sin v, (a + r \cos u) \cos v, 0)$ \\
\begin{align*}
  E &= r^2 (\sin^2 u \cos^2 v + \sin^2 u \sin^2 v + \cos^2 u) = r^2 \\
  G &= (a + r \cos u)^2 \sin^2 v + (a + r \cos u)^2 \cos^2 v = (a + r \cos u)^2 \\
  F &= r(a + r \cos u) \sin u \cos v \sin v - r (a + r \cos u) \sin u \sin v \cos v = 0
\end{align*}
\begin{align*}
  Area &= \int_{u=0}^{2\pi}\int_{v=0}^{2\pi} \sqrt{EG - F}\;du\;dv \\
       &= \int\int \sqrt{r^2(a + r \cosu)^2 - 0}\; du\; dv \\
       &= \int\int r(a + r \cosu)\;du\;dv \\
       &= \int\int ra\;du\;dv + \int\int r^2 \cos u\;du\;dv
       &= 2\pi(ra + (\sin 2\pi - \sin 0)) \\
       &= 2\pi(ra + (0 - 0)) \\
       &= 2\pi ra 
  \end{align*}
  

\underline{Why unit quaternion multiplicaiton represents a rotation in
  $\mathbb{R}^3$} \\

Let $H$ be $\mathbb{R}^4$.
Let $q \in S^3 = \{ q \in H : ||q||=1 \}$. For each $q$ dfeine a map $H \rightarrow H, x \mapsto q \times
\bar{q}$.
Note that $||q|| = 1\ implies ||q||^2 = 1 \implies q\bar{q} = 1 \implies q^{-1}
= \bar{q}$ \\
So $R_q(x) = q x \bar{q}$.
We claim $R_q$ is a rotation in $\mathbb{R}^3$. \\
To show $R_q$ is a rotation we show $R_q$ is linear and preserves the
the norm. First we show $R_q$ is a linear map. $R_q(cx + y) = q(cx + y)\bar{q} =
cqx\bar{q} + qy\bar{q} = cR_q(x) + R_q(y)$.\\
Now we show that $||R_q(x)|| = ||x||$,
\begin{align*}
  ||R_q(x)|| &= ||qx\bar{q}|| \\
             &= ||q||||x||||\bar{q}|| \\
             &= ||x||
\end{align*}
On the homework we will find matrix representation of $R_q$. It is
$[R_q(\vec{i}), R_q(\vec{j}), R_q(\vec{k})]$
\end{document}
