\documentclass[12pt,letterpaper]{hmcpset}
\usepackage[margin=1in]{geometry}
\DeclareMathOperator{\Tr}{Tr}
\usepackage{graphicx}
\usepackage{amsthm}
\usepackage{enumitem}
\usepackage{amsmath, cancel}

% Theorems
\usepackage{amsthm}
\renewcommand\qedsymbol{$\blacksquare$}
\makeatletter
\@ifclassloaded{article}{
    \newtheorem{definition}{Definition}[section]
    \newtheorem{example}{Example}[section]
    \newtheorem{theorem}{Theorem}[section]
    \newtheorem{corollary}{Corollary}[theorem]
    \newtheorem{lemma}{Lemma}[theorem]
}{
}
\makeatother

% Random Stuff
\setlength\unitlength{1mm}

\newcommand{\insertfig}[3]{
\begin{figure}[htbp]\begin{center}\begin{picture}(120,90)
\put(0,-5){\includegraphics[width=12cm,height=9cm,clip=]{#1}}\end{picture}\end{center}
\caption{#2}\label{#3}\end{figure}}

\newcommand{\insertxfig}[4]{
\begin{figure}[htbp]
\begin{center}
\leavevmode \centerline{\resizebox{#4\textwidth}{!}{\input
#1.pstex_t}}
\caption{#2} \label{#3}
\end{center}
\end{figure}}

\long\def\comment#1{}

\newcommand\abs[1]{\left\lvert#1\right\rvert}
\newcommand\norm[1]{\left\lVert#1\right\rVert}
\DeclareMathOperator*{\argmin}{arg\,min}
\DeclareMathOperator*{\argmax}{arg\,max}

% bb font symbols
\newfont{\bbb}{msbm10 scaled 700}
\newcommand{\CCC}{\mbox{\bbb C}}

\newfont{\bbf}{msbm10 scaled 1100}
\newcommand{\CC}{\mbox{\bbf C}}
\newcommand{\PP}{\mbox{\bbf P}}
\newcommand{\RR}{\mbox{\bbf R}}
\newcommand{\QQ}{\mbox{\bbf Q}}
\newcommand{\ZZ}{\mbox{\bbf Z}}
\renewcommand{\SS}{\mbox{\bbf S}}
\newcommand{\FF}{\mbox{\bbf F}}
\newcommand{\GG}{\mbox{\bbf G}}
\newcommand{\EE}{\mbox{\bbf E}}
\newcommand{\NN}{\mbox{\bbf N}}
\newcommand{\KK}{\mbox{\bbf K}}
\newcommand{\KL}{\mbox{\bbf KL}}

% Vectors
\renewcommand{\aa}{{\bf a}}
\newcommand{\bb}{{\bf b}}
\newcommand{\cc}{{\bf c}}
\newcommand{\dd}{{\bf d}}
\newcommand{\ee}{{\bf e}}
\newcommand{\ff}{{\bf f}}
\renewcommand{\gg}{{\bf g}}
\newcommand{\hh}{{\bf h}}
\newcommand{\ii}{{\bf i}}
\newcommand{\jj}{{\bf j}}
\newcommand{\kk}{{\bf k}}
\renewcommand{\ll}{{\bf l}}
\newcommand{\mm}{{\bf m}}
\newcommand{\nn}{{\bf n}}
\newcommand{\oo}{{\bf o}}
\newcommand{\pp}{{\bf p}}
\newcommand{\qq}{{\bf q}}
\newcommand{\rr}{{\bf r}}
\renewcommand{\ss}{{\bf s}}
\renewcommand{\tt}{{\bf t}}
\newcommand{\uu}{{\bf u}}
\newcommand{\ww}{{\bf w}}
\newcommand{\vv}{{\bf v}}
\newcommand{\xx}{{\bf x}}
\newcommand{\yy}{{\bf y}}
\newcommand{\zz}{{\bf z}}
\newcommand{\0}{{\bf 0}}
\newcommand{\1}{{\bf 1}}

% Matrices
\newcommand{\Ab}{{\bf A}}
\newcommand{\Bb}{{\bf B}}
\newcommand{\Cb}{{\bf C}}
\newcommand{\Db}{{\bf D}}
\newcommand{\Eb}{{\bf E}}
\newcommand{\Fb}{{\bf F}}
\newcommand{\Gb}{{\bf G}}
\newcommand{\Hb}{{\bf H}}
\newcommand{\Ib}{{\bf I}}
\newcommand{\Jb}{{\bf J}}
\newcommand{\Kb}{{\bf K}}
\newcommand{\Lb}{{\bf L}}
\newcommand{\Mb}{{\bf M}}
\newcommand{\Nb}{{\bf N}}
\newcommand{\Ob}{{\bf O}}
\newcommand{\Pb}{{\bf P}}
\newcommand{\Qb}{{\bf Q}}
\newcommand{\Rb}{{\bf R}}
\newcommand{\Sb}{{\bf S}}
\newcommand{\Tb}{{\bf T}}
\newcommand{\Ub}{{\bf U}}
\newcommand{\Wb}{{\bf W}}
\newcommand{\Vb}{{\bf V}}
\newcommand{\Xb}{{\bf X}}
\newcommand{\Yb}{{\bf Y}}
\newcommand{\Zb}{{\bf Z}}

% Calligraphic
\newcommand{\Ac}{{\cal A}}
\newcommand{\Bc}{{\cal B}}
\newcommand{\Cc}{{\cal C}}
\newcommand{\Dc}{{\cal D}}
\newcommand{\Ec}{{\cal E}}
\newcommand{\Fc}{{\cal F}}
\newcommand{\Gc}{{\cal G}}
\newcommand{\Hc}{{\cal H}}
\newcommand{\Ic}{{\cal I}}
\newcommand{\Jc}{{\cal J}}
\newcommand{\Kc}{{\cal K}}
\newcommand{\Lc}{{\cal L}}
\newcommand{\Mc}{{\cal M}}
\newcommand{\Nc}{{\cal N}}
\newcommand{\Oc}{{\cal O}}
\newcommand{\Pc}{{\cal P}}
\newcommand{\Qc}{{\cal Q}}
\newcommand{\Rc}{{\cal R}}
\newcommand{\Sc}{{\cal S}}
\newcommand{\Tc}{{\cal T}}
\newcommand{\Uc}{{\cal U}}
\newcommand{\Wc}{{\cal W}}
\newcommand{\Vc}{{\cal V}}
\newcommand{\Xc}{{\cal X}}
\newcommand{\Yc}{{\cal Y}}
\newcommand{\Zc}{{\cal Z}}

% Bold greek letters
\newcommand{\alphab}{\hbox{\boldmath$\alpha$}}
\newcommand{\betab}{\hbox{\boldmath$\beta$}}
\newcommand{\gammab}{\hbox{\boldmath$\gamma$}}
\newcommand{\deltab}{\hbox{\boldmath$\delta$}}
\newcommand{\etab}{\hbox{\boldmath$\eta$}}
\newcommand{\lambdab}{\hbox{\boldmath$\lambda$}}
\newcommand{\epsilonb}{\hbox{\boldmath$\epsilon$}}
\newcommand{\nub}{\hbox{\boldmath$\nu$}}
\newcommand{\mub}{\hbox{\boldmath$\mu$}}
\newcommand{\zetab}{\hbox{\boldmath$\zeta$}}
\newcommand{\phib}{\hbox{\boldmath$\phi$}}
\newcommand{\psib}{\hbox{\boldmath$\psi$}}
\newcommand{\thetab}{\hbox{\boldmath$\theta$}}
\newcommand{\taub}{\hbox{\boldmath$\tau$}}
\newcommand{\omegab}{\hbox{\boldmath$\omega$}}
\newcommand{\xib}{\hbox{\boldmath$\xi$}}
\newcommand{\sigmab}{\hbox{\boldmath$\sigma$}}
\newcommand{\pib}{\hbox{\boldmath$\pi$}}
\newcommand{\rhob}{\hbox{\boldmath$\rho$}}

\newcommand{\Gammab}{\hbox{\boldmath$\Gamma$}}
\newcommand{\Lambdab}{\hbox{\boldmath$\Lambda$}}
\newcommand{\Deltab}{\hbox{\boldmath$\Delta$}}
\newcommand{\Sigmab}{\hbox{\boldmath$\Sigma$}}
\newcommand{\Phib}{\hbox{\boldmath$\Phi$}}
\newcommand{\Pib}{\hbox{\boldmath$\Pi$}}
\newcommand{\Psib}{\hbox{\boldmath$\Psi$}}
\newcommand{\Thetab}{\hbox{\boldmath$\Theta$}}
\newcommand{\Omegab}{\hbox{\boldmath$\Omega$}}
\newcommand{\Xib}{\hbox{\boldmath$\Xi$}}

% mixed symbols
\newcommand{\sinc}{{\hbox{sinc}}}
\newcommand{\diag}{{\hbox{diag}}}
\renewcommand{\det}{{\hbox{det}}}
\newcommand{\trace}{{\hbox{tr}}}
\newcommand{\tr}{\trace}
\newcommand{\sign}{{\hbox{sign}}}
\renewcommand{\arg}{{\hbox{arg}}}
\newcommand{\var}{{\hbox{var}}}
\newcommand{\cov}{{\hbox{cov}}}
\renewcommand{\Re}{{\rm Re}}
\renewcommand{\Im}{{\rm Im}}
\newcommand{\eqdef}{\stackrel{\Delta}{=}}
\newcommand{\defines}{{\,\,\stackrel{\scriptscriptstyle \bigtriangleup}{=}\,\,}}
\newcommand{\<}{\left\langle}
\renewcommand{\>}{\right\rangle}
\newcommand{\Psf}{{\sf P}}
\newcommand{\T}{\top}
\newcommand{\m}[1]{\begin{bmatrix} #1 \end{bmatrix}}


% info for header block in upper right hand corner
\name{Joseph Gardi}
\class{Differential Geometry}
\assignment{Notes}
\duedate{Monday, Nov 4th 2019}

\renewcommand{\labelenumi}{{(\alph{enumi})}}


\begin{document}
\underline{Definition} A diffeomorphism $\phi: S \rightarrow \bar{S}$ is an isometry if for
all $p \in S$ and all pairs $w_1, w_2 \in T_p(S)$ we have,
\begin{align*}
  <w_1, w_2>_p = <d\phi_p(w_1), d\phi_p(w_2)>_{\phi(p)}
\end{align*}
$S$ and $\bar{S}$ are isometric. \\
The surface of a cylindar and the plane have the same first fundamental form. So they are
isometric.  \\

Let $\bar{u} = u, \bar{v} = \bar{a} \sinh v$.  \\
Let $x(\bar{u}, \bar{v}) = (\bar{v} \cos \bar{u}, \bar{v} \sin \bar{u}, a
\bar{v}) = (\bar{a} \sinh v \cos u, \bar{a} \sinh v \sin u, au)$. \\
Then $x_u = (-\bar{a} \sinh v \sin u, \bar{a} \sinh v \cos u, \bar{a})$, \\
$x_v = (\bar{a} \cosh v \cos u, \bar{a} \cosh v \sin u, 0)$. \\
Then $E = \bar{a}^2 \sinh^2 v + \bar{a}^2 = \bar{a}^2\cosh^2 v, F = 0, G =
\bar{a}^2 \cosh^2 v$.\\
\underline{Gauss Map}
Gauss Map is a map $S \rightarrow S^2 (\text{a unit sphere}), p \mapsto N(p)$ where $N(p)$ is the
unit normal vector to the surface at point $p$.\\
\underline{Self adjoint:} A linear map $L: V^n \rightarrow V^n$ is self adjoint if $<L(v),
w> = <v, L(w)>$. A matrix is self adjoint if and only if it is symetric. \\
\begin{enumerate}
\item $ll_p(v) \triangleq $ Normal curvature along $v$
\item $k_1, k_2 \triangleq$ Principal curvaure
\item $\det\; dN_p \triangleq$ gaussian curvature
\item $-\frac{1}{2} tr(dN_p) \triangleq$ mean curvature
\end{enumerate}
\underline{Theorem: } The differential $dN_p$ is a map from the tangent plane on
the sphere to the tangent plane on the surface. The planes are parallel. So
$dN_p(x_u) = N_u, dN_p(x_v) = N_v$. \\
The normal vector is orthogonal to $x_u, x_v$. Therefore,
\begin{align*}
  <N, x_u> &= 0 \\
  <N, x_v> &= 0 \\
  \implies <N_v, x_u> + <N, x_{uv}> &= 0 \\
  <N_u, x_v> + <N, x_{vu}> &= 0 \\
  \implies <N_u, x_v> &= <N_v, x_u> \\
  \implies <dN_p(x_u), x_v> = <x_u, dN_p(x_v)>
\end{align*}
Since $dN_p$ and the inner product are linear we only have to show it holds true
for basis vectors $x_u, x_v$.  \\
\underline{The second fundamental form: } The fact that $dN_p: T_p(S) \rightarrow T_p(S)$
is a self adjoint means we can associate to $dN_p$ a quadratic form $Q$ in
$T_p(S)$ given by $Q(v) = <dN_pv, v>, v \in T_p(S)$. To obtain a geometic
interpretation of the equadratic form, we need  a few definitions. For rerasons
that will soon become clear, we shall use the quadratic fomr $Q$. \\
\underline{Definition of second fundamental form:} It is the quadratic form
$ll_p$ defined in $T_p(S)$ by $ll_p(v) = -<dN_pv, v>$. \\
Let $C$ be a regular cuve in $S$ passing through $p \in S$,. Let $k$ be the
 curvature of $C$ at $p$ and let $\cos \theta = <N, n>$. $k_n = k \cos \theta$ is the
 normal curvature of $C$ at $p$. \\
 \underline{Pf of Euler's formula:} Normal cuvature along a given direction in
 $T_p(S)$ can be calculated from the principal curvaturees, $k_n = k_1 \cos^2 \theta
 + k_2 \sin^2 \theta$.  \\
 $k_n = -<dN_p(v), v> = -<dN_p(\cos \theta e_1 + \sin \theta e_2), v> = -<\cos \theta dN_p(e_1)
 + \sin \theta dN_p(e_2), v> = <k_1 \cos \theta e_1 + k_2 \sin \theta e_2, \cos \theta e_1 + \sin \theta
 e_2> = k_1\cos^2 \theta + k_2 \sin^2 \theta$. \\
 \underline{Gaussian curvature: } $k_1k_2$ \\
 \underline{Mean curvature: } $\frac{k_1 + k_2}{2}$ \\
 A point is elliptic if $\det dN_p > 0$, hyperbolic if $\det dN_p < 0$,
 parabolic if $\det dN_p = 0$ with $dN_p \neq 0$, and planar if $dN_p = 0$.
\end{document}
