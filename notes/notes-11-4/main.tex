\documentclass[12pt,letterpaper]{hmcpset}
\usepackage[margin=1in]{geometry}
\DeclareMathOperator{\Tr}{Tr}
\usepackage{graphicx}
\usepackage{amsthm}
\usepackage{enumitem}
\usepackage{amsmath, cancel}

\input{macros.tex}

% info for header block in upper right hand corner
\name{Joseph Gardi}
\class{Differential Geometry}
\assignment{Notes}
\duedate{Monday, Nov 4th 2019}

\renewcommand{\labelenumi}{{(\alph{enumi})}}


\begin{document}
\underline{Definition} A diffeomorphism $\phi: S \rightarrow \bar{S}$ is an isometry if for
all $p \in S$ and all pairs $w_1, w_2 \in T_p(S)$ we have,
\begin{align*}
  <w_1, w_2>_p = <d\phi_p(w_1), d\phi_p(w_2)>_{\phi(p)}
\end{align*}
$S$ and $\bar{S}$ are isometric. \\
The surface of a cylindar and the plane have the same first fundamental form. So they are
isometric.  \\

Let $\bar{u} = u, \bar{v} = \bar{a} \sinh v$.  \\
Let $x(\bar{u}, \bar{v}) = (\bar{v} \cos \bar{u}, \bar{v} \sin \bar{u}, a
\bar{v}) = (\bar{a} \sinh v \cos u, \bar{a} \sinh v \sin u, au)$. \\
Then $x_u = (-\bar{a} \sinh v \sin u, \bar{a} \sinh v \cos u, \bar{a})$, \\
$x_v = (\bar{a} \cosh v \cos u, \bar{a} \cosh v \sin u, 0)$. \\
Then $E = \bar{a}^2 \sinh^2 v + \bar{a}^2 = \bar{a}^2\cosh^2 v, F = 0, G =
\bar{a}^2 \cosh^2 v$.\\
\underline{Gauss Map}
Gauss Map is a map $S \rightarrow S^2 (\text{a unit sphere}), p \mapsto N(p)$ where $N(p)$ is the
unit normal vector to the surface at point $p$.\\
\underline{Self adjoint:} A linear map $L: V^n \rightarrow V^n$ is self adjoint if $<L(v),
w> = <v, L(w)>$. A matrix is self adjoint if and only if it is symetric. \\
\begin{enumerate}
\item $ll_p(v) \triangleq $ Normal curvature along $v$
\item $k_1, k_2 \triangleq$ Principal curvaure
\item $\det\; dN_p \triangleq$ gaussian curvature
\item $-\frac{1}{2} tr(dN_p) \triangleq$ mean curvature
\end{enumerate}
\underline{Theorem: } The differential $dN_p$ is a map from the tangent plane on
the sphere to the tangent plane on the surface. The planes are parallel. So
$dN_p(x_u) = N_u, dN_p(x_v) = N_v$. \\
The normal vector is orthogonal to $x_u, x_v$. Therefore,
\begin{align*}
  <N, x_u> &= 0 \\
  <N, x_v> &= 0 \\
  \implies <N_v, x_u> + <N, x_{uv}> &= 0 \\
  <N_u, x_v> + <N, x_{vu}> &= 0 \\
  \implies <N_u, x_v> &= <N_v, x_u> \\
  \implies <dN_p(x_u), x_v> = <x_u, dN_p(x_v)>
\end{align*}
Since $dN_p$ and the inner product are linear we only have to show it holds true
for basis vectors $x_u, x_v$.  \\
\underline{The second fundamental form: } The fact that $dN_p: T_p(S) \rightarrow T_p(S)$
is a self adjoint means we can associate to $dN_p$ a quadratic form $Q$ in
$T_p(S)$ given by $Q(v) = <dN_pv, v>, v \in T_p(S)$. To obtain a geometic
interpretation of the equadratic form, we need  a few definitions. For rerasons
that will soon become clear, we shall use the quadratic fomr $Q$. \\
\underline{Definition of second fundamental form:} It is the quadratic form
$ll_p$ defined in $T_p(S)$ by $ll_p(v) = -<dN_pv, v>$. \\
Let $C$ be a regular cuve in $S$ passing through $p \in S$,. Let $k$ be the
 curvature of $C$ at $p$ and let $\cos \theta = <N, n>$. $k_n = k \cos \theta$ is the
 normal curvature of $C$ at $p$. \\
 \underline{Pf of Euler's formula:} Normal cuvature along a given direction in
 $T_p(S)$ can be calculated from the principal curvaturees, $k_n = k_1 \cos^2 \theta
 + k_2 \sin^2 \theta$.  \\
 $k_n = -<dN_p(v), v> = -<dN_p(\cos \theta e_1 + \sin \theta e_2), v> = -<\cos \theta dN_p(e_1)
 + \sin \theta dN_p(e_2), v> = <k_1 \cos \theta e_1 + k_2 \sin \theta e_2, \cos \theta e_1 + \sin \theta
 e_2> = k_1\cos^2 \theta + k_2 \sin^2 \theta$. \\
 \underline{Gaussian curvature: } $k_1k_2$ \\
 \underline{Mean curvature: } $\frac{k_1 + k_2}{2}$ \\
 A point is elliptic if $\det dN_p > 0$, hyperbolic if $\det dN_p < 0$,
 parabolic if $\det dN_p = 0$ with $dN_p \neq 0$, and planar if $dN_p = 0$.
\end{document}
