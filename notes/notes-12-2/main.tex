\documentclass[12pt,letterpaper]{hmcpset}
\usepackage[margin=1in]{geometry}
\DeclareMathOperator{\Tr}{Tr}
\usepackage{graphicx}
\usepackage{amsthm}
\usepackage{enumitem}
\usepackage{amsmath, cancel}

\input{macros.tex}

% info for header block in upper right hand corner
\name{Joseph Gardi}
\class{Differential Geometry}
\assignment{Notes}
\duedate{Monday, Dec 12th 2019}

\renewcommand{\labelenumi}{{(\alph{enumi})}}

\begin{document}
\underline{Today}
\begin{enumerate}
\item  Covariant derivative
\item Parallel transport
\item Geodesic
\item G-B Theorem
\end{enumerate}
Intrinsic geometry means we are looking at just surfaces.
It lies in this coordinate. \\
$w$ is a vector field on a surface. $w(u, v) = a(u, v)x_u + b(u, v)x_v$ \\
Let's restrict $w$ onto the curve $\alpha(t) = (u(t), v(t))$. Then $w(t) = a(\alpha(t))x_u + b(\alpha(t))x_v$ and
\begin{align*}
  w'(t) &= a'(\alpha(t)) \alpha'(t) x_u\\& + a(\alpha(t))(x_{uu}u'(t) + x_{vv}v'(t))\\& + b'(\alpha(t)) 
  \alpha'(t) x_v\\& + b(\alpha(t))(x_{vu}u'(t) + x_{vv}v'(t)) \\
        &= qx_u + rx_v + sN &\tag{for some $q, r, s \in R$}
\end{align*}
If $Dw/dt = 0$ then that means the vector field is paralell to $\alpha'(t)$.\\
A vector field $w$ along a parameterized curve $\alpha: I \rightarrow S$ is said to be parallel
if $Dw/dt = 0$ for every $t \in I$. \\
Parallel transport in $R^2$ means  all the vectors are parallel.\\
\underline{Proposition} If $w, v$ are parallel vector fields then $<w(t), v(t)>$
is constant. \\
\underline{Definition} a nonconstant parameterized curve $\gamma: I \rightarrow S$ is said to
be geodesic at $t \in I$ if the field of its tangent vectors $\gamma'(t)$ is parallel
along $\gamma$ at $t$. That is,
\begin{align*}
  \frac{D\gamma'(t)}{dt} = 0
\end{align*}
$\gamma$ is a parameterized geodesic if it is geodesic for all $t \in I$. \\
\underline{Gauss-Bonnet} \\
\begin{align*}
  \int\int_S k\; ds = 2\pi \chi(S)
\end{align*}
$\chi$ is the euler number for the surface. $\chi(S) = #vertices - #edges + #faces$ \\

\end{document}
