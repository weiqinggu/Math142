\documentclass{article}
\usepackage{ulem}
\normalem
\usepackage{marvosym}
\usepackage{amssymb}
\usepackage{mathrsfs}
\usepackage{bm}
\usepackage{url,graphicx,tabularx,array,geometry}
%\usepackage{wasysym}
\usepackage{textcomp}
\usepackage{amsmath}

\topmargin=0.0in
\oddsidemargin=0.0in
\evensidemargin=0.0in
\textwidth=6.5in
\marginparwidth=0.5in
\headheight=0pt
\headsep=0pt
\textheight=9.0in

\newcommand{\zmod}[1]{\mathbb{Z}/#1\mathbb{Z}}

\renewcommand{\title}[1]{\textbf{#1}\\}
\renewcommand{\line}{\begin{tabularx}{\textwidth}{X>{\raggedleft}X}\hline\\\end{tabularx}\\[-0.5cm]}
\newcommand{\leftright}[2]{\begin{tabularx}{\textwidth}{X>{\raggedleft}X}#1%
& #2\\\end{tabularx}\\[-0.5cm]}

%\linespread{2} %-- Uncomment for Double Space
\begin{document}

\leftright{Differential Geometry- Homework \#2}{\today} %-- left and right positions in the header


% Problem2 - Part A
%----------------------------------------------------------------------------

\section*{Section 1-4 -- Problem 2}

\noindent A plane $P$ in $\mathbb{R}^3$ is given by $ax + by+ cz + d = 0$.  Show that the vector $v = (a,b,c)$ is perpendicular to the plane and that $\frac{|d|}{\sqrt{a^2 + b^2+c^2}}$ measure the distance from the plane to the origin.

\subsection*{Proof}

\noindent Consider two arbitrary points on the plane $(u_1, v_1, w_1)$ and $(u_2, v_2, w_2)$.  The vector going from point to the other is given by

\begin{equation*}
(u_2 - u_1, v_2-v_1, w_2 - w_1). \\
\end{equation*}

\noindent Using the dot product of this vector with $(a,b,c)$ we observe that 

\begin{eqnarray*}
(a,b,c)\cdot (u_2 - u_1, v_2-v_1 , w_2 - w_1) & = & (au_2 + bv_2 + cw_2) - (au_1 + bv_1 + cw_1) \\
& = & -d + d \;\; = \;\; 0. \\
\end{eqnarray*}

\noindent Therefore $(a,b,c)$ is perpendicular to all vectors lying in the plane $P$.  Therefore $(a,b,c)$ is perpendicular to $P$.  Suppose now that $(x_1, y_1, z_1)$ is the point on the plane that is the shortest distance from the origin.  The vector $(x_1, y_1, z_1)$ is perpendicular to the plane, and therefore must be parallel to $(a,b,c)$.  Thus, consider the unit vector $\frac{(a,b,c)}{\sqrt{a^2+ b^2 + c^2}}$ and take the dot product with this and $(x_1, y_1, z_1)$.  We obtain

\begin{eqnarray*}
\frac{(x_1, y_1, z_1) \cdot  (a,b,c)}{\sqrt{a^2 + b^2+c^2}} & = & \frac{ax_1 + by_1 + cz_1}{\sqrt{a^2 + b^2+c^2}} \\
& = & \frac{-d}{\sqrt{a^2+b^2+c^2}}. \\
\end{eqnarray*}

\noindent The distance $D$ from the origin to the plane is just the absolute value of the quantity above. 

\begin{equation*}
D \;\; = \;\; \frac{|d|}{\sqrt{a^2 + b^2+c^2}}. \\
\end{equation*}

\noindent \hfill $\Box$

% Section 1-4 -- Problem 11
%-----------------------------------------------------------------------------

\section*{Section 1-4 -- Problem 11}

\noindent Show that the volume $V$ of a parallelepiped generated by three linearly independent vectors $u,v,w \in \mathbb{R}^3$ is given by $V = |(u \times v) \cdot w|$ and introduce an oriented volume in $\mathbb{R}^3$.  Subsequently, prove that

\begin{equation*}
V^2 \;\; = \;\; \left | \begin{array}{ccc}
u \cdot u & u \cdot v & u \cdot w \\
v \cdot u & v\cdot v & v\cdot w \\
w \cdot u & w \cdot v & w\cdot w \\
\end{array} \right |. \\
\end{equation*}

\subsection*{Proof}

\noindent Consider the vectors $u$ and $v$ and the parallelogram generated by these two vectors.  The area of this parallelogram is given by $||u|| \; ||v|| \sin \theta$ where $\theta$ is the angle between $u$ and $v$.  We recognize this as the familiar cross-product in $\mathbb{R}^3$, hence the area of this parallelogram is $||u \times v||$.  The volume of the parallelepiped is the area of this parallelogram times a distance perpendicular to the plane containing $u$ and $v$.  Notice that since $w$ is linearly independent from $u$ and $v$, that there is a component of $w$ that is perpendicular to both $u$ and $v$, and hence either parallel or anti-parallel to $u \times v$.  Therefore, the volume of the parallelepiped is given by $|(u \times v) \cdot w|$.  Observe that we can interpret the volume as the determinant of a matrix given by 

\begin{equation*}
V \;\; = \;\; \left | \begin{array}{ccc}
w_1 & w_2 & w_3 \\
u_1 & u_2 & u_3 \\
v_1 & v_2 & v_3 \\
\end{array} \right |. \\
\end{equation*}

\noindent Observe that $V$ does not change if we perform an equal number of row exchanges on the above matrix.  First exchanging the first and second rows, and subsequently the second and third rows we obtain

\begin{equation*}
V \;\; = \;\; \left | \begin{array}{ccc}
u_1 & u_2 & u_3 \\
v_1 & v_2 & v_3 \\
w_1 & w_2 & w_3 \\
\end{array} \right |. \\
\end{equation*}

\noindent If the above matrix is given by $A$, we know that $V^2 = \det(A)^2 = \det(A)\det(A) = \det(A)\det(A^T) = \det(AA^T)$.  We therefore write

\begin{eqnarray*}
AA^T &  = & \left [ \begin{array}{ccc}
u_1 & u_2 & u_3 \\
v_1 & v_2 & v_3 \\
w_1 & w_2 & w_3 \\
\end{array} \right] \left [\begin{array}{ccc}
u_1 & v_1 & w_1 \\
u_2 & v_2 & w_2 \\
u_3 & v_3 & w_3 \\
\end{array} \right ] \\
& = & \left [ \begin{array}{ccc}
u_1^2 + u_2^2 + u_3^2 & u_1v_1 + u_2v_2 + u_3v_3  & u_1w_1 + u_2w_2 + u_3w_3 \\
v_1u_1 + v_2u_2 + v_3u_3 & v_1^2 + v_2^2 + v_3^2 & v_1w_1 + v_2w_2 + v_3w_3 \\
w_1u_1 + w_2u_2 + w_3u_3 & w_1v_1 + w_2v_2 + w_3v_3 & w_1^2 + w_2^2 + w_3^2 \\
\end{array} \right ] \\
& = & \left [ \begin{array}{ccc}
u\cdot u & u\cdot v & u\cdot w \\
v\cdot u & v\cdot v & v\cdot w \\
w\cdot u & w\cdot v & w\cdot w \\
\end{array} \right ]. \\
\end{eqnarray*}

\noindent Therefore

\begin{equation*}
V^2 \;\; = \;\; \left | \begin{array}{ccc}
u\cdot u & u\cdot v & u\cdot w \\
v\cdot u & v\cdot v & v\cdot w \\
w\cdot u & w\cdot v & w\cdot w \\
\end{array} \right |. \\
\end{equation*}

\noindent \hfill $\Box$

% Section B -- Problem 1
%-----------------------------------------------------------------------------

\section*{Section B -- Problem 1}

\noindent Find the length of the curve obtain by intersecting the sphere $x^2 +y^2 + z^2 =4$ and the cylinder $(x-1)^2 + y^2 = 1$ in $\mathbb{R}^3$.  

\subsection*{Solution}

\noindent We can first parametrize the cylinder by taking $x(t) = 1+ \cos t$ and $y(t) = \sin t$ for $t \in [0, 2\pi]$.  Observe this satisfies the identity $\cos^2t + \sin^2t = 1$.  Next we obtain an equation of a parametrized curve in $\mathbb{R}^3$ by substituting these values of $x(t)$ and $y(t)$ into the equation of the sphere and solving for $z(t)$.  We see that

\begin{eqnarray*}
(1+ \cos t)^2 + \sin^2t + z^2(t) & = & 4 \\
\cos^2t + \sin^2t + 2\cos t + 1 + z^2(t) & = & 4 \\
2 + 2\cos t + z^2(t) & = & 4 \\
z^2(t) & = & 2(1 - \cos t).
\end{eqnarray*}

\noindent We now find the arclength of the curve in the first octant.  Observe that at $t= 0$ we have $y(0) = 0$, and hence this is the smallest value of $t$ such that the curve lies in the first octant.  At $t = \pi$ we have $z^2(\pi) = 4$ and hence $t$ obtains its largest value in the first octant at $t = \pi$.  The equation of the curve is then 

\begin{equation*}
\alpha(t) \;\; = \;\;\left (1+ \cos t, \sin t, \sqrt{2(1 - \cos t)} \right ) \hspace{2pc} t \in [0, \pi]. \\
\end{equation*}

\noindent To find the arclength we first find the derivative $\alpha'(t)$.

\begin{equation*}
\alpha'(t) \;\; = \;\; \left ( - \sin t, \cos t, \frac{\sin t}{\sqrt{2(1 - \cos t)}} \right ). \\
\end{equation*}

\noindent Now find the norm of $\alpha'(t)$.

\begin{eqnarray*}
||\alpha'(t)||^2 & = & \sin^2t + \cos^2t + \frac{\sin^2t}{2(1 - \cos t)} \\
& = & 1 + \frac{1 - \cos^2t}{2(1 - \cos t)} \\
&  = & 1 + \frac{(1- \cos t)(1+ \cos t)}{2(1 - \cos t)} \\ 
& =  & \frac{1}{2}(3 + \cos t). \\
\end{eqnarray*}

\noindent Therefore the arclength of this curve in the first octant is 

\begin{eqnarray*}
\int_0^\pi ||\alpha'(t)|| \; dt & = & \int_0^\pi \sqrt{ \frac{3 + \cos t}{2}} \; dt \\
& \approx & 3.8202. \\
\end{eqnarray*}

\noindent Integration was done numerically on Wolfram Alpha.  \hfill $\Box$

% Section B -- Problem 2
%-----------------------------------------------------------------------------

\section*{Section B -- Problem 2}

\noindent Write in detail the proof from class: Assume that all the normal vectors of a parametrized curve pass through the same fixed point.  Prove that the trace of the curve is contained in a circle.  

\subsection*{Proof}

\noindent Let $\alpha: [a,b] \to \mathbb{R}^3$ be given such that for all $s \in [a,b]$ we have that $n(s)$ passes through the same point $p$.  Assuming without loss of generality that $p=0$, let $r(t)$ be the equation of a line given by $r(\lambda) = \alpha(s) + \lambda n(s)$, which is the line passing through $\alpha(s)$ coinciding with the normal $n(s)$.  Since every normal vector passes through the point $p$, there exists $\lambda_o \in \mathbb{R}$ such that $r(\lambda_o) = p = 0 = \alpha(s) + \lambda_o n(s)$.  Taking the derivative of both sides of this equation, we obtain

\begin{eqnarray*}
0 & = & \alpha'(s) + \lambda_o'(s) n(s) + \lambda_o(s) n'(s) \\
& = & t(s) + \lambda_o'(s)n(s) +  \lambda_o(s)(-\kappa(s) t(s) - \tau(s) b(s)) \\
& = & (1 - \lambda_o(s) \kappa(s)) t(s) + \lambda_o'(s) n(s) - \lambda_o(s)\tau(s) b(s), \\
\end{eqnarray*}

\noindent where $\kappa(s)$ and $\tau(s)$ are the curvature and torsion of $\alpha(t)$, respectively, and $t(s), n(s)$, and $b(s)$ are the tangent, normal, and binormal vectors respectively.  Notice that $t(s), n(s)$, and $b(s)$ form the moving Frenet frame of the curve $\alpha(t)$, which by construction is an orthonormal basis.  Therefore, the only way the above equation is zero is if the coefficients of each of the unit vectors is identically zero.  Therefore the coefficient of the $n(s)$ satisfies $\lambda_o'(s) = 0$ implying that $\lambda_o(s)$ is constant, which we define as $\lambda_o(s) = \lambda_o$.  For the coefficient of the binormal vector $b(s)$ we have $\lambda_o \tau(s) = 0$.  We know that $\lambda_o \neq 0$, since if we recall the equation $r(\lambda) = \alpha(s) + \lambda n(s)$, at $\lambda = \lambda_o$ this would imply $\alpha(s) = 0$ for all $s \in [a,b]$, a contradiction since $\alpha(s)$ is a curve.  Therefore $\lambda_o \neq 0$, implying the torsion $\tau(s) = 0$ for all $s \in [a,b]$.  Lastly observe that $1 - \lambda_o \kappa(s) = 0$, implying that $\kappa(s) = \frac{1}{\lambda_o}$, a constant.  Since the torsion is zero and the curvature is constant, this implies that $\alpha(s)$ lies in a plane and the constant curvature implies that the trace of $\alpha(s)$ is either the arc of a circle, or a circle itself.  \hfill $\Box$

% Section 1-2 -- Problem 4
%-----------------------------------------------------------------------------

\section*{Section 1-2 -- Problem 4}

\noindent Let $\alpha: I \to \mathbb{R}^3$ be a parametrized curve and let $v \in \mathbb{R}^3$ be a  fixed vector.  Assume that $\alpha'(t)$ is orthogonal to $v$ for all $t \in I$ and that $\alpha(0)$ is also orthogonal to $v$.  Prove that $\alpha(t)$ is orthogonal to $v$ for all $t \in I$.  

\subsection*{Proof}

\noindent Let $g(t)$ be the function given by $\alpha(t) \cdot v = g(t)$.  Taking the derivative of both sides we obtain $\alpha'(t) \cdot v = g'(t)$.  We know however that since $\alpha'(t)$ is orthogonal to $v$ for all $t \in I$ that $g'(t) = 0$.  Therefore $g(t) = c$ where $c$ is a constant.  Hence $\alpha(t) \cdot v = c$ is constant for all $t \in I$.  Notice however that we are given that $\alpha(0) \cdot v = 0$.  This initial condition implies that $c = 0$.  Therefore $\alpha(t) \cdot v = 0$ for all $t \in I$, hence $\alpha(t)$ is orthogonal to $v$ for all $t \in I$.  \hfill $\Box$

% Section 1-2 -- Problem 5
%-----------------------------------------------------------------------------

\section*{Section 1-2 -- Problem 5}

\noindent Let $\alpha: I \to \mathbb{R}^3$ be a parametrized curve with $\alpha'(t) \neq 0$ for all $t \in I$.  Show that $||\alpha(t)||$ is a nonzero constant if and only if $\alpha(t)$ is orthogonal to $\alpha'(t)$ for all $t \in I$.  

\subsection*{Proof}

\noindent $\Longrightarrow$

\noindent Suppose $||\alpha(t)|| = c$ for all $t \in I$ where $c$ is a nonzero constant.  We therefore have that $\alpha(t) \cdot \alpha(t) = c^2$.  Taking the derivative of both sides we obtain

\begin{equation*}
\alpha'(t) \cdot \alpha(t) + \alpha(t) \cdot \alpha'(t) \;\; = \;\; 2\alpha'(t)\cdot \alpha(t) \;\; = \;\; 0
\end{equation*}

\noindent implying that $\alpha'(t)\cdot \alpha(t) = 0$ for all $t \in I$. 

\vspace{1pc}

\noindent $\Longleftarrow$

\noindent Suppose $\alpha(t) \cdot \alpha'(t) = 0$ for all $t \in I$.  Consider the norm $||\alpha(t)||$.  Taking the derivative of this we obtain

\begin{equation*}
\frac{d}{dt} ||\alpha(t)||^2 \;\; = \;\; \alpha(t)\cdot \alpha'(t) + \alpha'(t)\cdot \alpha(t) \;\; = \;\; 2 \alpha(t) \cdot \alpha'(t) \;\; = \;\;0. \\
\end{equation*}

\noindent This implies that $||\alpha(t)|| = c$ where $c$ is a constant.  We know that $c \neq 0$, since this would imply that the curve $\alpha(t) \equiv 0$, which is a contradiction since it violates the definition of a parametrized curve.  \hfill $\Box$

% Section 1-2 -- Problem 1
%-----------------------------------------------------------------------------

\section*{Section 1-2 -- Problem 1}

\noindent Given the parametrized curve (helix)

\begin{equation*}
\alpha(s) \;\; = \;\; \left ( a \cos \frac{s}{c}, \; a \sin \frac{s}{c}, \; b \frac{s}{c} \right )
\end{equation*}

\noindent for $s \in \mathbb{R}$ and $c^2 = a^2 + b^2$.  

\begin{enumerate}
\item Show that the parameter $s$ is the arclength.
\item Determine the curvature and the torsion of $\alpha$.
\item Determine the osculating plane of $\alpha$.
\item Show that the lines containing $n(s)$ and passing through $\alpha(s)$ meet the $z$-axis under a constant angle equal to $\frac{\pi}{2}$.
\item Show that the tangent lines to $\alpha$ make a constant angle with the $z$-axis.  
\end{enumerate}

%%%%%%%% Part 1
\subsection*{Proof Part 1}

\noindent Observe the derivative

\begin{equation*}
\alpha'(s) \;\; = \;\; \left ( -\frac{a}{c} \sin \frac{s}{c}, \; \frac{a}{c} \cos \frac{s}{c}, \; \frac{b}{c} \right ). \\
\end{equation*}

\noindent Finding the norm $||\alpha'(s)||^2$ we find that 

\begin{eqnarray*}
||\alpha'(s)||^2 & = & \frac{a^2}{c^2} \sin^2\left ( \frac{s}{c} \right ) + \frac{a^2}{c^2} \cos^2 \left ( \frac{s}{c} \right ) + \frac{b^2}{c^2} \\
& = & \frac{a^2 + b^2}{c^2} \;\; = \;\;1
\end{eqnarray*}

\noindent since we are given that $c^2 = a^2 + b^2$.  Therefore $||\alpha'(s)|| = 1$, implying that $s$ is the arclength.  \hfill $\Box$

%%%%%%%% Part 2

\subsection*{Proof Part 2}

\noindent Recall that the curvature $\kappa(s)$ is given by the equation $t'(s) = \kappa(s)n(s)$.  We know that $\alpha'(s) = t(s)$.  Taking another derivative yields

\begin{equation*}
\alpha''(s) \;\; = \;\; t'(s) \;\; = \;\; \left ( - \frac{a}{c^2} \cos \frac{s}{c}, \; - \frac{a}{c^2} \sin \frac{s}{c}, \; 0 \right ) \;\; = \;\; - \frac{a}{c^2} \left ( \cos \frac{s}{c}, \; \sin \frac{s}{c}, \; 0 \right ). \\
\end{equation*}

\noindent A simple computation shows that $||\alpha''(s)|| = \frac{a}{c^2}$.  Observe that since

\begin{equation*}
n(s) \;\; = \;\; \frac{\alpha''(s)}{||\alpha''(s)||} \;\; = \;\; - \left ( \cos \frac{s}{c}, \; \sin \frac{s}{c}, \; 0 \right ). \\
\end{equation*}

\noindent This implies that $t'(s) = \frac{a}{c^2} n(s)$ demonstrating that $\kappa(s) = \frac{a}{c^2}$, so the curvature is constant.  To find the torsion, we first find the binormal vector $b(s) = t(s) \times n(s)$.  We can calculate this by finding the determinant

\begin{eqnarray*}
b(s) & = & \left | \begin{array}{ccc}
\hat{\textbf{x}} & \hat{\textbf{y}} & \hat{\textbf{z}} \\
- \frac{a}{c} \sin \frac{s}{c} & \frac{a}{c} \cos \frac{s}{c} & \frac{b}{c} \\
- \cos \frac{s}{c} & - \sin \frac{s}{c} & 0 \\
\end{array} \right | \\
& = & \left ( \frac{b}{c} \sin \frac{s}{c}, \; - \frac{b}{c} \cos \frac{s}{c}, \; \frac{a}{c} \right ). \\
\end{eqnarray*}

\noindent We know that $b'(s) = \tau(s) n(s)$, hence when we take the derivative we find that

\begin{eqnarray*}
b'(s) & =  & \left ( - \frac{b}{c} \cos \frac{s}{c}, \; - \frac{b}{c} \sin \frac{s}{c}, \; 0 \right ) \\
& = & \frac{b}{c} \left ( - \cos \frac{s}{c}, \; - \sin \frac{s}{c} , \; 0 \right ) \\
& = & \frac{b}{c} n(s)
\end{eqnarray*}

\noindent implying that $\tau(s) =\frac{b}{c}$, so the torsion is constant. \hfill $\Box$

%%%%%%%%% Part 3

\subsection*{Proof Part 3}

\noindent The osculating plane is the plane spanned by $t(s)$ and $n(s)$.  By definition we have $b(s)$ normal to the osculating plane.  Therefore the dot product between $b(s)$ and any vector lying in the plane is zero.  Let $(x,y,z)$ be a point in the osculating plane.  Since the normal vector lies in the osculating plane, we have the vector $ \left ( x + \cos \frac{s}{c}, y + \sin \frac{s}{c}, z \right )$ lies in the osculating plane.  We therefore write

\begin{eqnarray*}
b(s) \cdot  \left ( x + \cos \frac{s}{c}, y + \sin \frac{s}{c}, z \right )& = & \left ( \frac{b}{c} \sin \frac{s}{c}, \; - \frac{b}{c} \cos \frac{s}{c}, \; \frac{a}{c} \right ) \cdot  \left ( x + \cos \frac{s}{c}, y + \sin \frac{s}{c}, z \right ) \\
& = & x \frac{b}{c} \sin \frac{s}{c} - y \frac{b}{c} \cos \frac{s}{c} + \frac{b}{c} \left ( \sin \frac{s}{c} \cos\frac{s}{c} - \sin \frac{s}{c} \cos \frac{s}{c} \right ) +z \frac{a}{c}  \\
& = & x \sin \frac{s}{c} - y \cos \frac{s}{c} + \frac{a}{b}z \;\; = \;\; 0. \\
\end{eqnarray*}

\noindent Written more succinctly, the equation of the osculating plane is 

\begin{equation*}
x \sin \left (\frac{s}{c} \right )- y \cos \left ( \frac{s}{c} \right )+ \frac{a}{b} z \;\; = \;\;0.
\end{equation*}

\noindent \hfill $\Box$

%%%%%%%% Part 4

\subsection*{Proof Part 4}

\noindent For all $s \in \mathbb{R}$ let $r_s(t)$ be the line passing through $\alpha(s)$ and coinciding with $n(s)$, so that we can write $r_s(t) = \alpha(s) +t n(s)$.  We can write this as 

\begin{equation*}
r_s(t) \;\; = \;\; \left ( (a-t) \cos \frac{s}{c}, \; (a-t) \sin \frac{s}{c}, \; b\frac{s}{c} \right ). \\
\end{equation*}

\noindent Therefore, when $t =a$ we have $r_s(a) = \left (0,0, b \frac{s}{c} \right )$.  Observe that the derivative $r'_s(t)$ is given by $r'_s(t) = n(s)$.  Finding the dot product $r_s(a) \cdot n(s)$ we observe that

\begin{equation*}
r_s(a) \cdot n(s) \;\; = \;\; \left ( 0, 0, b \frac{s}{c} \right ) \cdot \left ( -\cos \frac{s}{c}, \; -\sin \frac{s}{c}, \; 0 \right ) \;\; =\;\;0.
\end{equation*}

\noindent This implies that the line $r_s(t)$ meets the $z$-axis perpendicularly, at an angle of $\frac{\pi}{2}$.  \hfill $\Box$

%%%%%%% Part 5
\subsection*{Proof Part 5}

\noindent We now show that the tangent lines to $\alpha(s)$ make a constant angle with the $z$-axis.  This is equivalent to showing that the tangent vectors make a constant angle to the $z$-axis.  For all $s \in \mathbb{R}^3$ consider $\alpha'(s)$.  We can observe that given the unit vector $(0,0,1)$ that 

\begin{equation*}
\alpha'(s) \cdot (0,0,1) \;\; = \;\; \frac{b}{c}.  
\end{equation*} 

\noindent Similarly, the inner product is given by $\alpha'(s) \cdot (0,0,1) = ||\alpha'(s)|| \; ||(0,0,1)|| \cos \theta$.  We know however that $||\alpha'(s)|| = ||(0,0,1)|| = 1$, hence equating the two values of the inner product, we obtain that $\cos\theta = \frac{b}{c}$.  This proves that the angle made between each tangent vector and the $z$-axis is constant.  \hfill $\Box$

% Section 1-2 -- Problem 2
%-----------------------------------------------------------------------------

\section*{Section 1-2 -- Problem 2}

\noindent Show that the torsion $\tau(s)$ of $\alpha(s)$ is given by 

\begin{equation*}
\tau(s) \;\; = \;\; - \frac{(\alpha'(s) \times \alpha''(s)) \cdot \alpha'''(s)}{|\kappa(s)|^2}. \\
\end{equation*}

\subsection*{Proof}

\noindent Recall that $n'(s) = -\kappa(s) t(s) - \tau(s) b(s)$.  Using the inner product on both sides with $b(s)$ we obtain that $\tau(s) = - b(s) \cdot n'(s)$.  Remember that $b(s) = t(s) \times n(s)$.  Hence we obtain

\begin{equation*}
\tau(s) \;\; = \;\; - (t(s) \times n(s)) \cdot n'(s) \;\; = \;\; - \frac{(\alpha'(s) \times \alpha''(s)) \cdot n'(s)}{\kappa(s)}. \\
\end{equation*}

\noindent Since we know that $n(s) = \frac{\alpha''(s)}{\kappa(s)}$ we can take the derivative to obtain

\begin{equation*}
n'(s) \;\; =\;\; \frac{\alpha'''(s)}{\kappa(s)} - \frac{\kappa'(s) \alpha''(s)}{\kappa^2(s)}. \\
\end{equation*}

\noindent Note that the vector $\alpha'(s) \times \alpha''(s)$ is perpendicular to $\alpha''(s)$, hence $\alpha'(s) \times \alpha''(s)$ is perpendicular to the second term in the expression above for $n'(s)$.  We can therefore ignore this second term and write the expression for $\tau(s)$

\begin{equation*}
\tau(s) \;\; = \;\; - \frac{(\alpha'(s) \times \alpha''(s)) \cdot \alpha'''(s)}{\kappa^2(s)}, \\
\end{equation*}

\noindent as claimed. \hfill $\Box$

% Section 1-2 -- Problem 5
%-----------------------------------------------------------------------------

\section*{Section 1-2 -- Problem 5}

\noindent A regular parametrized curve has the property that all its tangent lines pass through a fixed point.

\begin{enumerate}
\item Prove that the trace of $\alpha$ is (a segment of) a straight line.  
\item Does the conclusion in part 1 hold if $\alpha$ is not regular?
\end{enumerate}

\subsection*{Proof Part 1}

\noindent For every point $s$ for which $\alpha(s)$ is defined, let $r_s(\lambda) = \alpha(s) + \lambda t(s)$.  By hypothesis we know that there exists $\lambda_o$ and some fixed point $p$ such that for all $s$ where $\alpha(s)$ is defined, $r_s(\lambda_o) = p = \alpha(s) + \lambda_ot(s)$.  Without loss of generality we can let $p=0$, hence $\alpha(s) + \lambda_ot(s) = 0$.  Taking the derivative of both sides with respect to $s$ yields $\alpha'(s) + \lambda_o'(s) t(s) + \lambda_o(s) t'(s) = 0$.  This can be simplified as 

\begin{eqnarray*}
0 & = & t(s) + \lambda_o'(s) t(s) + \lambda_o(s) \kappa(s) n(s) \\
& = & (1 + \lambda_o'(s)) t(s) + \lambda_o(s)\kappa(s) n(s). \\
\end{eqnarray*}

\noindent Since $t(s)$ and $n(s)$ make up a subspace of the Frenet frame, we have that $t(s)$ and $n(s)$ are orthogonal, hence the only way the above equation is equal to zero is if each of the coefficients is equal to 0.  This implies that $\lambda_o'(s) = -1$, hence $\lambda_o(s) = -s + c$ where $c$ is some constant.  Similarly, we have that $\lambda_o(s) \kappa(s) = 0$.  However, since we know that $\lambda_o(s)$ varies linearly with respect to $s$ and $\lambda_o(s)\kappa(s)$ must be zero for all $s$, this implies that $\kappa(s) \equiv 0$.  Thus the curvature of $\alpha(s)$ is everywhere 0.  This implies that $t(s)$ is constant, hence $\alpha(s)$ is either a straight line or a segment of a straight line in $\mathbb{R}^3$.  \hfill $\Box$

%%%%%%%%% Proof Part 2

\subsection*{Proof Part 2}

\noindent Let $\alpha(s)$ be a straight line in $\mathbb{R}^3$, and hence have a constant tangent vector $t(s)$.  Suppose $\alpha(s)$ is not regular, implying that for some $s_o$ on which $\alpha(s)$ is defined we have $t(s_o) = 0$.  However, since $t(s)$ is constant, this would imply $t(s) \equiv 0$, further implying that $\alpha(s)$ is constant for all $s$.  Therefore $\alpha(s)$ would not be a curve, but simply would be a point in $\mathbb{R}^3$, a contradiction.  Therefore a straight line in $\mathbb{R}^3$ is regular.  Taking the contrapositive of this statement, we see that a non-regular curve in $\mathbb{R}^3$ cannot be a straight line.  \hfill $\Box$































\end{document}






