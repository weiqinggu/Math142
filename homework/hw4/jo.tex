
% HMC Math dept HW template example
% v0.04 by Eric J. Malm, 10 Mar 2005
\documentclass[12pt,letterpaper,boxed]{hmcpset}

% set 1-inch margins in the document
\usepackage[margin=1in]{geometry}

% include this if you want to import graphics files with /includegraphics
\usepackage{graphicx,siunitx,cancel,scrextend,amsthm, amssymb, enumerate,
  hyperref, parskip, mathtools, xcolor, amsfonts, xcolor}

\input{macros.tex}

\DeclareSIUnit\year{yr}
\newcommand{\inden}{\begin{addmargin}[2em]{0em}}
\renewcommand{\mod}[3]{#1 \equiv #2 \ (\mbox{mod } #3)}
\newcommand{\R}{\mathbb{R}}
\newcommand{\Mod}[1]{\ (\mathrm{mod}\ #1)}

\newcommand{\highlight}[1]{%
  \colorbox{red!50}{$\displaystyle#1$}}

\newcommand{\cmmt}[1]{%
  \text{\phantom{(#1)}} \tag{#1}
}

\DeclarePairedDelimiter\ceil{\lceil}{\rceil}
\DeclarePairedDelimiter\floor{\lfloor}{\rfloor}

\name{Joseph Gardi}					%% insert your name
\class{Math142} 		%% replace spaces (\ \ \ ) with section #
\assignment{Homework 4}
\duedate{Due: Monday, Octoboer 7, 2019}

\begin{document}
\paragraph{A: Problems on Reviewing of Rigid Motions in $R^3$.}
\begin{itemize}

{\item a) Show that the set of rigid motions $E(3)$ forms a group. 
(Later, we will see that  $E(3)$ is in fact a Lie group.)} 
 
\end{itemize}
\begin{solution}
  
\end{solution}
\medskip
\paragraph{B: Problems from Lectures}

\begin{itemize}
{\item a) Show that of all simple closed curves in the plane with 
given length $l$, a circle bounds the largest area.}
\begin{solution}
See The isoperimetric inequality on Do Carmo page 33.
\end{solution}
\end{itemize}

\paragraph{C: Other Problems}
\begin{itemize}
{\item a) Problem 2 on page 29, Section 1-6, Baby Do Carmo.}
\begin{solution}
  The osculating plane is the unique plane containing $\alpha(s), \alpha(s) + \alpha'(s), \alpha(s)
  + \alpha''(s)$. Let $P_{h_1, h_2}$ be the plane containing $\alpha(s), \alpha(s + h_1), \alpha(s +
  h_2)$.  It is given that $\alpha(s) \in P_{h_1, h_2}$.\\
  Now we show $\alpha(s) + \alpha'(s) \in P_{h_1, h_2}$. All affine combinations of those points are contained in $P_{h_1, h_2}$ so
  $\alpha(s) + \alpha'(s) = \alpha(s) + \frac{1}{h_1}(\alpha(s + h_1) - \alpha(s)) \in P_{h_1, h_2}$. 
  Now we show $\alpha(s) + \alpha''(s) \in P_{h_1, h_2}$.
  \begin{align*}
    \alpha(s) + \alpha''(s) &= \alpha(s) + \frac{1}{h_2}(\alpha'(s + h_2) - \alpha'(s)) \\
                  &= \alpha(s) + \frac{1}{h_2}(\frac{\alpha(s + h_2) - \alpha(s + h_1)}{h_2 - h_1} - \alpha'(s)) \\
                  &\in P_{h_1, h_2} \cmmt{Since this is an affine combination of poitns in the plane} \\
  \end{align*}
\end{solution}
{\item b) Problem 1 on page 47, Section 1-7, Baby Do Carmo.}
\begin{solution}
  No. That would violate the isoperimetric inequality. \\\\
\end{solution}
{\item c) Problem 2 on page 47, Section 1-7, Baby Do Carmo.}
\begin{solution}
Suppose that we have a curve $E$ of length $l$ from $A$ to $B$ that is part of a larger circle $D$ with length $g$. We know from the isoperimetric
inequality that this circle is the closed cuve of length $g$ that bounds the
largest possible area. If there was a curve $C$ of length $l$ from $A$ to $B$ that together
with $\overline{AB}$ bounds a larger area than $E$ with $\overline{AB}$ that would
contradict the isoperimetric theorem because that would imply that replaceing
$E$ with $C$ in the circle $D$ would create a shape with length $g$ that bounds
more area than the circle $D$.
\end{solution}
{\item d) Problem 3 on page 65, Section 2-2, Baby Do Carmo.}
{\item e) Problem 5 on page 65, Section 2-2, Baby Do Carmo.}
{\item f) Problem 10 on page 66, Section 2-2, Baby Do Carmo.}
{\item g) Problem 16 on page 67, Section 2-2, Baby Do Carmo.}

\end{itemize}

\paragraph{D: Extra Credit Problems}
\begin{itemize}
{\item Give a different solution to B a).}
\end{itemize}

\end{document}