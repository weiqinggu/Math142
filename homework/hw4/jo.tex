
% HMC Math dept HW template example
% v0.04 by Eric J. Malm, 10 Mar 2005
\documentclass[12pt,letterpaper,boxed]{hmcpset}

% set 1-inch margins in the document
\usepackage[margin=1in]{geometry}

% include this if you want to import graphics files with /includegraphics
\usepackage{graphicx,siunitx,cancel,scrextend,amsthm, amssymb, enumerate,
  hyperref, parskip, mathtools, xcolor, amsfonts, xcolor}

\input{macros.tex}

\DeclareSIUnit\year{yr}
\newcommand{\inden}{\begin{addmargin}[2em]{0em}}
\renewcommand{\mod}[3]{#1 \equiv #2 \ (\mbox{mod } #3)}
\newcommand{\R}{\mathbb{R}}
\newcommand{\Mod}[1]{\ (\mathrm{mod}\ #1)}

\newcommand{\highlight}[1]{%
  \colorbox{red!50}{$\displaystyle#1$}}

\newcommand{\cmmt}[1]{%
  \text{\phantom{(#1)}} \tag{#1}
}

\DeclarePairedDelimiter\ceil{\lceil}{\rceil}
\DeclarePairedDelimiter\floor{\lfloor}{\rfloor}

\name{Joseph Gardi}					%% insert your name
\class{Math142} 		%% replace spaces (\ \ \ ) with section #
\assignment{Homework 4}
\duedate{Due: Monday, Octoboer 7, 2019}

\begin{document}
\paragraph{A: Problems on Reviewing of Rigid Motions in $R^3$.}
\begin{itemize}

{\item a) Show that the set of rigid motions $E(3)$ forms a group. 
(Later, we will see that  $E(3)$ is in fact a Lie group.)} 
 
\end{itemize}

\begin{solution}
 For this problem, I referenced an explanation of $E(3)$ given in a PDF by John Baez on the UCR \href{http://math.ucr.edu/home/baez/classical/galilei_pro.pdf}{Classical Mechanics} website.

The set of rigid motions $E(3)$ contains all pairs $(R, t)$ such that $R \in 0(3)$ is an orthogonal transformation (a rotation) and $t \in \RR^3$  is a translation vector. Each element $(R, t)$ gives a transformation of $3$-dimensional Euclidean space built from an orthogonal transformation and a translation:
\[
f_{(R, t)} \colon \RR^3 \to \RR^3
\]
defined by 
\[
f_{(R, t)}(x) = Rx + t
\]


Recall that a set is a group if it is equipped with a binary operation that satisfies the axioms of closure, associativity, identity, and invertibility.

\textbf{Closure}

Given elements $(R, t) , (R', t') \in E(3)$, the composition of the transformations is 
\begin{align*}
    f_{(R, t)} \circ f_{(R', t')}(x) 
    &= R (R' x + t') + t\\
    &= RR'x + Rt' + t.
\end{align*}
Since $RR' \in O(3)$ and $Rt' + t \in \RR^n$, the composed transformation is also in $E(3)$ and thus $E(3)$ is closed under composition. 

\textbf{Associativity}

We assert that, given elements $(R,t), (R',t'), (R'', t'') \in E(3)$, then
\[
f_{(R, t)} \circ \Big(f_{(R', t')} \circ f_{(R', t')}\Big) = \Big(f_{(R, t)} \circ f_{(R', t')}\Big) \circ f_{(R', t')}.
\]
\begin{proof}
As a function of $x$, the left hand side of the above composition is given by
\begin{align*}
   f_{(R, t)} \circ \Big(f_{(R', t')} \circ f_{(R', t')}\Big)(x)
   &= R (R' (R''x + t'') + t') + t\\
   &= R(R'R'')x + R(R't'' + t') + t \\
   &= (RR')R''x + (RR')t'' + (Rt' + t)\\
   &= \Big(f_{(R, t)} \circ f_{(R', t')}\Big) \circ f_{(R', t')}(x)
\end{align*}
\end{proof}

\textbf{Identity}

The pair $(I_3, 0) \in E(3)$ is the identity element. The proof is left as an exercise to the grader.

\textbf{Invertibility}

Any element $(R, t)$ in $E(3)$ has an inverse $(R^T, -R^Tt)$ in $E(3)$.

\begin{proof}
\begin{align*}
    f_{(R, t)} \circ f_{(R^T, -t)} 
    &= f_{(RR^T, -RR^Tt + t)} \\
    &= f_{(I_3, 0)}.
\end{align*}
Similarly,
\begin{align*}
    f_{(R^T, R^Tt)} \circ f_{(R, t)} 
    &= f_{(R^TR, R^TRt - t)} \\
    &= f_{(I_3, 0)}.
\end{align*}
As the composition of the two transformations has resulted in the identity element, the inverse exists and $(R^T, -R^Tt)$ is the proper inverse. 

\end{proof}

As all of the group axioms hold, $E(3)$ is a group.

 
\end{solution}
\medskip
\paragraph{B: Problems from Lectures}

\begin{itemize}
{\item a) Show that of all simple closed curves in the plane with 
given length $l$, a circle bounds the largest area.}
\begin{solution}
See The isoperimetric inequality on Do Carmo page 33.
\end{solution}
\end{itemize}

\paragraph{C: Other Problems}
\begin{itemize}
{\item a) Problem 2 on page 29, Section 1-6, Baby Do Carmo.}
\begin{solution}
  a) The osculating plane is the unique plane containing $\alpha(s), \alpha(s) + \alpha'(s), \alpha(s)
  + \alpha''(s)$. Let $P_{h_1, h_2}$ be the plane containing $\alpha(s), \alpha(s + h_1), \alpha(s +
  h_2)$.  It is given that $\alpha(s) \in P_{h_1, h_2}$.\\
  Now we show $\alpha(s) + \alpha'(s) \in P_{h_1, h_2}$. All affine combinations of those points are contained in $P_{h_1, h_2}$ so
  $\alpha(s) + \alpha'(s) = \alpha(s) + \frac{1}{h_1}(\alpha(s + h_1) - \alpha(s)) \in P_{h_1, h_2}$. 
  Now we show $\alpha(s) + \alpha''(s) \in P_{h_1, h_2}$.
  \begin{align*}
    \alpha(s) + \alpha''(s) &= \alpha(s) + \frac{1}{h_2}(\alpha'(s + h_2) - \alpha'(s)) \\
                  &= \alpha(s) + \frac{1}{h_2}(\frac{\alpha(s + h_2) - \alpha(s + h_1)}{h_2 - h_1} - \alpha'(s)) \\
                  &\in P_{h_1, h_2} \cmmt{Since this is an affine combination of poitns in the plane} \\
  \end{align*}
  b) Let $a$ be the center of this circle. Let $r$ be the radius so $r = ||\alpha(s)
  - a|| = ||\alpha(s + h_1) - a|| = ||\alpha(s + h_2) - a||$. We know that $a$ must lie in
  the osculating plane since we just showed in part a that $\alpha(s), \alpha(s + h_1),
  \alpha(s + h_2)$ all lie in the osculating plane. The line through $\alpha(s)$ and $\alpha(s
  + h_1)$ is tangent to the circle. $n(s)$  is in the osculating plane and
  orthogonal to the tangent line so it must be pointed towards the center of the
  circle. Let's make a pameterization for our circle and use the osculating
  plane as our coordinate system with the origin at $a$: $\beta(t) = (r cos
  \frac{t}{r}, r sin \frac{t}{r})$. This is already parameterized by arc length.
  The curvature is,
\begin{align*}
  ||\beta''(t)|| &= \sqrt{(-\frac{1}{r} cos \frac{t}{r})^2 +
    (-\frac{1}{r} sin \frac{t}{r})^2} \\
             &= \frac{1}{r} \sqrt{cos^2 \frac{t}{r} + sin^2 \frac{t}{r}} \\
             &= \frac{1}{r}
  \end{align*}
  The curvature of the circle $||\beta''(t)||$ is equal to the curvature of the
  given curve $||\alpha''(s)||$ because they share those 3 points. So we get
  $\frac{1}{r} = k(s) \implies r = \frac{1}{k(s)}$.
\end{solution}
{\item b) Problem 1 on page 47, Section 1-7, Baby Do Carmo.}
\begin{solution}
  No. That would violate the isoperimetric inequality. \\\\
\end{solution}
{\item c) Problem 2 on page 47, Section 1-7, Baby Do Carmo.}
\begin{solution}
Suppose that we have a curve $E$ of length $l$ from $A$ to $B$ that is part of a larger circle $D$ with length $g$. We know from the isoperimetric
inequality that this circle is the closed cuve of length $g$ that bounds the
largest possible area. If there was a curve $C$ of length $l$ from $A$ to $B$ that together
with $\overline{AB}$ bounds a larger area than $E$ with $\overline{AB}$ that would
contradict the isoperimetric theorem because that would imply that replaceing
$E$ with $C$ in the circle $D$ would create a shape with length $g$ that bounds
more area than the circle $D$.
\end{solution}
{\item d) Problem 3 on page 65, Section 2-2, Baby Do Carmo.}
\begin{solution}
 It was shown in the book that a one sheeted cone is not a regular surface. The
 double sheeted cone contains the one sheeted cone so it can't be a regular
 surface.
 It would still have the issue of not being a differentiable function in any
 form at $(0, 0, 0)$.
\end{solution}
{\item e) Problem 5 on page 65, Section 2-2, Baby Do Carmo.} \\
It is a parameterization. $x$ is surjective to to the neighborhood $V =
B_{.1}((1, 1, 0))$. 
{\item f) Problem 10 on page 66, Section 2-2, Baby Do Carmo.} \\
no. There is a ciritical point at the part where the loops meet.
{\item g) Problem 16 on page 67, Section 2-2, Baby Do Carmo.}\\
Given $u, v$ we want to find $\pi^{-1}(u, v)$.
We know the following
\begin{align*}
  ||\pi^{-1}(u, v) - (0, 0, 1)|| &= 1 \\
  \exists \alpha, (0, 0, 2) + \alpha(\pi^{-1}(u, v) - (0, 0, 2)) &= (u, v, 0) 
\end{align*}
Therefore,
\begin{align*}
  \pi^{-1}(u, v) &= \frac{1}{\alpha}(u, v, -2) + (0, 0, 2) \tag*{eq 1} \\
  ||\pi^{-1}(u, v) - (0, 0, 1)|| &= ||\frac{1}{\alpha}(u, v, -2) + (0, 0, 1)|| \\
               &= \sqrt{(u/\alpha)^2 + (v/\alpha)^2 + (1  - \frac{2}{\alpha})^2} = 1 \\
  \implies \frac{u^2 + v^2}{\alpha^2} + 1 - 4/\alpha + 4/\alpha^2 &= 1 \\
  \implies \frac{u^2 + v^2 + 4}{\alpha} - 4 &= 0 \\
  \implies \alpha &= \frac{u^2 + v^2 + 4}{4} \tag*{eq 2}
\end{align*}
By plugging in equation 2 for $\alpha$ into equation 1 for $\pi^{-1}(u, )$ we get the desired result
\end{itemize}

\paragraph{D: Extra Credit Problems}
\begin{itemize}
{\item Give a different solution to B a).}
\end{itemize}

\end{document}