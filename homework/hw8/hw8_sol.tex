\documentclass[12pt,letterpaper]{hmcpset}
\usepackage[margin=1in]{geometry}
\usepackage{graphicx}
\usepackage{amsthm}
\usepackage{enumitem}
\usepackage{amsmath}
\usepackage{changepage}
\usepackage{breqn}
\setlength{\parindent}{0 pt}
\setlength{\parskip}{1 em}

\usepackage{hyperref}
\hypersetup{
    colorlinks=true,
    linkcolor=blue,
    filecolor=magenta,      
    urlcolor=cyan,
}
 
% Theorems
\usepackage{amsthm}
\renewcommand\qedsymbol{$\blacksquare$}
\makeatletter
\@ifclassloaded{article}{
    \newtheorem{definition}{Definition}[section]
    \newtheorem{example}{Example}[section]
    \newtheorem{theorem}{Theorem}[section]
    \newtheorem{corollary}{Corollary}[theorem]
    \newtheorem{lemma}{Lemma}[theorem]
}{
}
\makeatother

% Random Stuff
\setlength\unitlength{1mm}

\newcommand{\insertfig}[3]{
\begin{figure}[htbp]\begin{center}\begin{picture}(120,90)
\put(0,-5){\includegraphics[width=12cm,height=9cm,clip=]{#1}}\end{picture}\end{center}
\caption{#2}\label{#3}\end{figure}}

\newcommand{\insertxfig}[4]{
\begin{figure}[htbp]
\begin{center}
\leavevmode \centerline{\resizebox{#4\textwidth}{!}{\input
#1.pstex_t}}
\caption{#2} \label{#3}
\end{center}
\end{figure}}

\long\def\comment#1{}

\newcommand\abs[1]{\left\lvert#1\right\rvert}
\newcommand\norm[1]{\left\lVert#1\right\rVert}
\DeclareMathOperator*{\argmin}{arg\,min}
\DeclareMathOperator*{\argmax}{arg\,max}

% bb font symbols
\newfont{\bbb}{msbm10 scaled 700}
\newcommand{\CCC}{\mbox{\bbb C}}

\newfont{\bbf}{msbm10 scaled 1100}
\newcommand{\CC}{\mbox{\bbf C}}
\newcommand{\PP}{\mbox{\bbf P}}
\newcommand{\RR}{\mbox{\bbf R}}
\newcommand{\QQ}{\mbox{\bbf Q}}
\newcommand{\ZZ}{\mbox{\bbf Z}}
\renewcommand{\SS}{\mbox{\bbf S}}
\newcommand{\FF}{\mbox{\bbf F}}
\newcommand{\GG}{\mbox{\bbf G}}
\newcommand{\EE}{\mbox{\bbf E}}
\newcommand{\NN}{\mbox{\bbf N}}
\newcommand{\KK}{\mbox{\bbf K}}
\newcommand{\KL}{\mbox{\bbf KL}}

% Vectors
\renewcommand{\aa}{{\bf a}}
\newcommand{\bb}{{\bf b}}
\newcommand{\cc}{{\bf c}}
\newcommand{\dd}{{\bf d}}
\newcommand{\ee}{{\bf e}}
\newcommand{\ff}{{\bf f}}
\renewcommand{\gg}{{\bf g}}
\newcommand{\hh}{{\bf h}}
\newcommand{\ii}{{\bf i}}
\newcommand{\jj}{{\bf j}}
\newcommand{\kk}{{\bf k}}
\renewcommand{\ll}{{\bf l}}
\newcommand{\mm}{{\bf m}}
\newcommand{\nn}{{\bf n}}
\newcommand{\oo}{{\bf o}}
\newcommand{\pp}{{\bf p}}
\newcommand{\qq}{{\bf q}}
\newcommand{\rr}{{\bf r}}
\renewcommand{\ss}{{\bf s}}
\renewcommand{\tt}{{\bf t}}
\newcommand{\uu}{{\bf u}}
\newcommand{\ww}{{\bf w}}
\newcommand{\vv}{{\bf v}}
\newcommand{\xx}{{\bf x}}
\newcommand{\yy}{{\bf y}}
\newcommand{\zz}{{\bf z}}
\newcommand{\0}{{\bf 0}}
\newcommand{\1}{{\bf 1}}

% Matrices
\newcommand{\Ab}{{\bf A}}
\newcommand{\Bb}{{\bf B}}
\newcommand{\Cb}{{\bf C}}
\newcommand{\Db}{{\bf D}}
\newcommand{\Eb}{{\bf E}}
\newcommand{\Fb}{{\bf F}}
\newcommand{\Gb}{{\bf G}}
\newcommand{\Hb}{{\bf H}}
\newcommand{\Ib}{{\bf I}}
\newcommand{\Jb}{{\bf J}}
\newcommand{\Kb}{{\bf K}}
\newcommand{\Lb}{{\bf L}}
\newcommand{\Mb}{{\bf M}}
\newcommand{\Nb}{{\bf N}}
\newcommand{\Ob}{{\bf O}}
\newcommand{\Pb}{{\bf P}}
\newcommand{\Qb}{{\bf Q}}
\newcommand{\Rb}{{\bf R}}
\newcommand{\Sb}{{\bf S}}
\newcommand{\Tb}{{\bf T}}
\newcommand{\Ub}{{\bf U}}
\newcommand{\Wb}{{\bf W}}
\newcommand{\Vb}{{\bf V}}
\newcommand{\Xb}{{\bf X}}
\newcommand{\Yb}{{\bf Y}}
\newcommand{\Zb}{{\bf Z}}

% Calligraphic
\newcommand{\Ac}{{\cal A}}
\newcommand{\Bc}{{\cal B}}
\newcommand{\Cc}{{\cal C}}
\newcommand{\Dc}{{\cal D}}
\newcommand{\Ec}{{\cal E}}
\newcommand{\Fc}{{\cal F}}
\newcommand{\Gc}{{\cal G}}
\newcommand{\Hc}{{\cal H}}
\newcommand{\Ic}{{\cal I}}
\newcommand{\Jc}{{\cal J}}
\newcommand{\Kc}{{\cal K}}
\newcommand{\Lc}{{\cal L}}
\newcommand{\Mc}{{\cal M}}
\newcommand{\Nc}{{\cal N}}
\newcommand{\Oc}{{\cal O}}
\newcommand{\Pc}{{\cal P}}
\newcommand{\Qc}{{\cal Q}}
\newcommand{\Rc}{{\cal R}}
\newcommand{\Sc}{{\cal S}}
\newcommand{\Tc}{{\cal T}}
\newcommand{\Uc}{{\cal U}}
\newcommand{\Wc}{{\cal W}}
\newcommand{\Vc}{{\cal V}}
\newcommand{\Xc}{{\cal X}}
\newcommand{\Yc}{{\cal Y}}
\newcommand{\Zc}{{\cal Z}}

% Bold greek letters
\newcommand{\alphab}{\hbox{\boldmath$\alpha$}}
\newcommand{\betab}{\hbox{\boldmath$\beta$}}
\newcommand{\gammab}{\hbox{\boldmath$\gamma$}}
\newcommand{\deltab}{\hbox{\boldmath$\delta$}}
\newcommand{\etab}{\hbox{\boldmath$\eta$}}
\newcommand{\lambdab}{\hbox{\boldmath$\lambda$}}
\newcommand{\epsilonb}{\hbox{\boldmath$\epsilon$}}
\newcommand{\nub}{\hbox{\boldmath$\nu$}}
\newcommand{\mub}{\hbox{\boldmath$\mu$}}
\newcommand{\zetab}{\hbox{\boldmath$\zeta$}}
\newcommand{\phib}{\hbox{\boldmath$\phi$}}
\newcommand{\psib}{\hbox{\boldmath$\psi$}}
\newcommand{\thetab}{\hbox{\boldmath$\theta$}}
\newcommand{\taub}{\hbox{\boldmath$\tau$}}
\newcommand{\omegab}{\hbox{\boldmath$\omega$}}
\newcommand{\xib}{\hbox{\boldmath$\xi$}}
\newcommand{\sigmab}{\hbox{\boldmath$\sigma$}}
\newcommand{\pib}{\hbox{\boldmath$\pi$}}
\newcommand{\rhob}{\hbox{\boldmath$\rho$}}

\newcommand{\Gammab}{\hbox{\boldmath$\Gamma$}}
\newcommand{\Lambdab}{\hbox{\boldmath$\Lambda$}}
\newcommand{\Deltab}{\hbox{\boldmath$\Delta$}}
\newcommand{\Sigmab}{\hbox{\boldmath$\Sigma$}}
\newcommand{\Phib}{\hbox{\boldmath$\Phi$}}
\newcommand{\Pib}{\hbox{\boldmath$\Pi$}}
\newcommand{\Psib}{\hbox{\boldmath$\Psi$}}
\newcommand{\Thetab}{\hbox{\boldmath$\Theta$}}
\newcommand{\Omegab}{\hbox{\boldmath$\Omega$}}
\newcommand{\Xib}{\hbox{\boldmath$\Xi$}}

% mixed symbols
\newcommand{\sinc}{{\hbox{sinc}}}
\newcommand{\diag}{{\hbox{diag}}}
\renewcommand{\det}{{\hbox{det}}}
\newcommand{\trace}{{\hbox{tr}}}
\newcommand{\tr}{\trace}
\newcommand{\sign}{{\hbox{sign}}}
\renewcommand{\arg}{{\hbox{arg}}}
\newcommand{\var}{{\hbox{var}}}
\newcommand{\cov}{{\hbox{cov}}}
\renewcommand{\Re}{{\rm Re}}
\renewcommand{\Im}{{\rm Im}}
\newcommand{\eqdef}{\stackrel{\Delta}{=}}
\newcommand{\defines}{{\,\,\stackrel{\scriptscriptstyle \bigtriangleup}{=}\,\,}}
\newcommand{\<}{\left\langle}
\renewcommand{\>}{\right\rangle}
\newcommand{\Psf}{{\sf P}}
\newcommand{\T}{\top}
\newcommand{\m}[1]{\begin{bmatrix} #1 \end{bmatrix}}


\DeclareMathOperator{\atan}{atan}
\DeclareMathOperator{\acos}{acos}
\DeclareMathOperator{\R}{\mathbb{R}}

% info for header block in upper right hand corner
\name{----------------------}
\class{Differential Geometry}
\assignment{Homework 8}
\duedate{Monday, November 18 2019}

\renewcommand{\labelenumi}{{(\alph{enumi})}}

\begin{document}

\begin{problem}[C.a) Problem 5 on page 168, Section 3-3, Baby Do Carmo.]
\\ \\
Consider the parametrized surface (Enneper’s surface)  
$$x(u,v) = \left(u-\frac{u^3}{3}+ uv^2,v-\frac{v^3}{3}+ vu^2, u^2-v^2 \right)$$
Show that 
\begin{enumerate}
    \item The coefficients of the first fundamental form are
    $$E=G=(1+u^2+v^2)^2, \quad F=0$$
    \item The coefficients of the second fundamental form are 
    $$e = 2,\quad g=-2,\quad f=0$$
    \item The principal curvatures are  
    $$k_1 = \frac{2}{(1+u^2+v^2)^2}, \quad k_2 = \frac{-2}{(1+u^2+v^2)^2}$$\\
\end{enumerate}
\end{problem}
\begin{solution}
\end{solution}

\begin{problem}[C.b) Problem 1 on page 185, Section 3-4, Baby Do Carmo.]
\\ \\
Prove that the differentiability of a vector field does not depend on the choice of a coordinate system.\\
\end{problem}
\begin{solution}
\end{solution}

\begin{problem}[C.c) Problem 2 on page 185, Section 3-4, Baby Do Carmo.]
\\ \\
Prove that the vector field obtained on the torus by parametrizing all its meridians by arc length and taking their tangent vectors (Example 1) is differentiable.\\
\end{problem}
\begin{solution}
\end{solution}

\begin{problem}[C.d) Problem 5 on page 186, Section 3-4, Baby Do Carmo.]
\\ \\
Let $S$ be a surface and let $x:U\to S$ be a parametrization of $S$. If $ac-b^2 <0$, show that 
$$a(u,v)(u')^2+2b(u,v)u'v'+c(u,v)(v')^2=0$$
can be factored into two distinct equations, each of which determines a field of directions on $X(U)\subset S$. Prove that these two fields of directions are orthogonal if and only if 
$$Ec-2Fb+Ga = 0$$\\
\end{problem}
\begin{solution}
\end{solution}

\begin{problem}[C.e) Problem 8 on page 187, Section 3-4, Baby Do Carmo.]
\\ \\
Show that if $w$ is a differentiable vector field on a surface $S$ and $w(p)\neq 0$ for some $p\in S$, then it is possible to parametrize a neighborhood of $p$ by $x(u,v)$ in such a way that $x_u = w$. 
\end{problem}
\begin{solution}
\end{solution}

\newpage 
\textbf{Extra Credit Problems} \\
\\
\begin{problem}[D.a) Problem 10 on page 187, Section 3-4, Baby Do Carmo.]
\\ \\
Let $T$ be the Torus of Example 6 of Sec. 2-2 and define a map $\phi:\R^2\to T$ by 
$$\phi(u,v) = ((r\cos u+a)\cos v, (r\cos u + a)\sin v, r\sin u),$$
where $u$ and $v$ are the Cartesian coordinates of $\R^2$. Let $u=at, v=bt$ be a straight line in $\R^2$, passing by $(0,0) \in \R^2$, and consider the curve in $T$ $\alpha(t) = \phi(at,bt)$. Prove that 
\begin{enumerate}
    \item $\phi$ is a local diffeomorphism. 
    \item The curve $\alpha(t)$ is a regular curve; $\alpha(t)$ is a closed curve if and only if $b/a$ is a rational number.
    \item If $b/a$ is irrational, the curve $\alpha(t)$ is dense in $T$; that is, in each neighborhood of a point $p\in T$ there exists a point of $\alpha(t)$\\
\end{enumerate}
\end{problem}
\begin{solution}
\end{solution}

\begin{problem}[D.b) Problem 11 on page 187, Section 3-4, Baby Do Carmo.]
\\ \\
Use the local uniqueness of trajectories of a vector field $w$ in $U\subset S$ to prove the following result. Given $p\in U$, there exists a unique trajectory $\alpha:I\to U$ of $w$, with $\alpha(0) = p$, which is maximal in the following sense: Any other trajectory $\beta:J\to U$, with $\beta(0) = p$, is the restriction of $\alpha$ to $J$ (i.e $J\subset I$ and $\alpha|_J = \beta$)\\
\end{problem}
\begin{solution}
\end{solution}

\begin{problem}[D.c) Problem 12 on page 187, Section 3-4, Baby Do Carmo.]
\\ \\
Prove that if $w$ is a differentiable vector field on a compact surface $S$ and $\alpha(t)$ is the maximal trajectory of $w$ with $\alpha(0) = p \in S$, then $\alpha(t)$ is defined for all $t\in \R$. 
\end{problem}
\begin{solution}
\end{solution}


\end{document}









